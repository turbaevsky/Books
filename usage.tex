\section{Performance Indicator Usage}
\subsection{Specific Uses}
\subsubsection{Individual Unit Values }

As described in the available Reference Manual, a primary purpose of the performance
indicators is to monitor an individual unit’s performance.

This indicator can be trended over a period of time to note improving
or declining performance.  The period values can be as brief as
quarterly values, which provide the most recent results but can jump from
quarter to quarter.  The values can be as long as 3-year values, which will provide the
smoothest curves and indicate long-term  performance.  Most WANO indicator
definitions are defined as 3-year values.

Individual performance use is summarized below:

Individual units:
\begin{itemize}
\item Trending
\item Comparison
  \begin{itemize}
  \item Between individuals
  \item Individual : group
  \end{itemize}
\item Goal setting
\end{itemize}

\subsubsection{Group Indicator Values }

Another primary purpose of the performance indicators is to monitor a group’s
performance.  This group can be pre-defined, as in WANO’s PI REPORTS application,
or any group of units defined by a user.  Similar to individual unit values, group values
(best quartile, median, worst quartile, mean) can be trended over a period of time to note
improving or declining performance.

Group performance use is summarized below:
\begin{itemize}
\item Trending
  \begin{itemize}
  \item Utility performance
  \item Member performance
  \item Select grouping performance
  \end{itemize}
\item Comparison
  \begin{itemize}
  \item Between groups
  \item Individual : group
  \end{itemize}
\item Goal setting
\end{itemize}

\subsubsection{Composite Indicator Values (Index) }

Some members or groups use WANO performance indicators to provide a measure of
overall unit performance by combining several indicators into one valu
e.  This value is usually defined on a 100 point basis and referred
 to as the performance indicator index, or simply, the index.

Points are assigned per indicator using the calculated indicator value to determine the
number of points.  Each indicator’s points are then assigned a weighting factor such that
the sum of points can be a maximum of 100 or a minimum of zero.  A group median and
quartiles are then obtained using the individual unit values within the group.  This
method is discussed in more detail in the index section on the WANO performance
indicator Web pages.

The unit index may be trended to indicate a change in overall performance.  The
individual unit performance can also be compared to a group’s quartile values to provide
an indication of relative performance to other units within the group.  Ranking of
individual units is NOT recommended due to reactor plant differences and other factors
that are not comparable.

Group values may be trended for indication of overall group performance, but content of
the group and number of units within the group must be considered.

\paragraph{Examples of Currently-Used Composite Indicators:}

\begin{itemize}
\item INPO PI Index 2007 (not accessible to PI REPORTS due to use of non-WANO data)
\item INPO PI Index 2006 (Method 4)
\item INPO PI Index 2003 (Method 3)
\end{itemize}
