\section*{Introduction}
\subsection*{Purpose}
This manual addresses the administrative management of the World
Association of Nuclear Operators (WANO) Performance Indicator (PI)
programme. The purpose of this manual is to ensure WANO performance
indicator data is collected, submitted, processed and used
consistently within WANO.

The Performance Indicator programme is implemented in accordance with
WANO Programme Guideline WPG 04, Technical Support \& Exchange. A
complete process description with programme requirements is provided
in the PCD 2014-2, Performance Indicator Process Description.

\subsection*{Background}
The WANO performance indicators have been adopted to provide a
quantitative indication of plant performance in the areas of nuclear
plant safety and reliability and personnel safety. These indicators
are intended principally for use by nuclear operating organisations to
monitor performance and progress, to set challenging goals for
improvement, to gain additional perspective on performance relative to
that of other plants, and to provide an indication of the possible
need to adjust priorities and resources to achieve improved overall
performance. WANO performance indicators are intended to support the
exchange of operating experience information and to allow consistent
comparisons of nuclear plant performance. It is expected that WANO
performance indicators will encourage emulation of the best industry
performance and motivate the identification and exchange of good
practices in nuclear plant operation.

These indicator descriptions were developed through the combined
efforts of the Institute of Nuclear Power Operations (INPO), the
International Union of Producers and Distributors of Electrical Energy
(UNIPEDE), WANO, and the respective members and participants of these
organisations, to develop a uniform set of performance indicators for
international use. The performance indicators provided in this
document were developed to be as simple as possible, while still
providing useful and meaningful trends and comparisons of performance
among nuclear power plants.

WANO performance indicators are established on one or more of the
following criteria:
\begin{itemize}
\item The indicator provides a quantitative indication of nuclear safety, plant reliability and personnel safety.
\item The programme is limited to a few indicators that monitor fundamental results rather than the performance of intermediate processes or individual programme elements.
\item The indicator has wide applicability.
\item The indicator provides a meaningful perspective without a detailed knowledge of plant programmes and practices.
\item The indicator is objective and fair.
\item The indicator is amenable to goal setting.
\item Data is available and reliable.
\item Emphasis on improving the indicator value is unlikely to cause undesirable plant actions.
\item Indicators that primarily monitor plant reliability should reflect performance only in areas that can be controlled or influenced by plant management.
\item Indicators of nuclear plant or personnel safety should reflect
  overall plant performance including, in some cases, elements beyond
  plant management control
\end{itemize}
The indicators have received extensive international review and constitute a uniform set of indicators suitable for international use. Revisions to this manual incorporate experience gained in the use of the programme.
\subsection*{Data Collection and Storage}
Each WANO member nuclear operating organisation is responsible for the
collection of quarterly data elements identified in the data entry
system section of the WANO website. This data should be collected for
each operating (commercial operation) nuclear plant/reactor unit, and
provided to the member's regional centre. All significant source data
changes should be commented in the appropriate PI DES fields.

Regional centres review the submitted data, resolve data anomalies or reporting problems with the members and promote the data to the production server within 60 days following the end of each calendar month. The data submitted by the regional centres will be entered into the performance indicator database. It is important to note that the database will only accept quarterly performance indicator data elements. The database is then used to calculate the indicator values. Calculated indicator values cannot be submitted by operating organisations.
Database has all data including unit current status information, design specification, source date, indicators, history etc. for all members’ units, particularly including units under construction and decommissioning period.
PI report includes only result for units in commercial operation.

All changes in design-related information like Reference Unit Power (RUP) etc. should be sent from operating organisation/plant to regional centre, and then from regional centres to WANO London Office (LO). Submitted data is reviewed by RC and LO by comparing with LO Core DB and IAEA PRIS. Discrepancies are to resolve continuously.
As noted above, regional centres are responsible for reviewing data
submitted by the operating organisations to ensure there are no
obvious reporting errors, and the data is consistent with the WANO
performance indicator definitions. Regional centres will process
submitted data as soon as possible. All PI data submitted must be
through the data entry system of the WANO website.

In order to present performance indicators representative of operating plants, regional centres should officially inform WANO London Office (LO) and Atlanta Centre (WANO AC) of units that are shut down for decommissioning and those entering commercial service. Indicator results from decommissioned (permanently shut down) units will be removed from the database beginning with the first calendar quarter after the unit’s last full quarter of reported data.
In addition, regional centres should evaluate units in long-term
shutdown (LTS) and consider excluding performance data, associated
from the LTS units from the database. The decision about changing unit
status can be taken only by RC director after appropriate unit’s
request, if one or more of the following circumstances apply:
\begin{itemize}
\item restart is inhibited primarily by political or institutional factors and not by the ability of the plant to operate safely
\item a restart date has not been firmly established
\item restart is not being aggressively pursued
\item other unusual and non-technical factors affecting a unit's
  performance that would distort the industry value
\end{itemize}
Then RC director or RC responsible person officially informs WANO LO
(Chief Executive Officer (CEO) and/or Technical Support and Exchange
programme (TSE) director) for applying of necessary changes within
Databases (DBs).

When a unit restarts after a long-term shutdown, its data should be
requalified for use in the industry-wide calculations beginning with
the calendar quarter following return to commercial operation or full
power. Data for units entering commercial operation is first included
in the performance indicator database in the first full quarter of
operation following commencement of commercial operation. Commercial
operation commences when all required commissioning testing is
completed.

\subsection*{Report Development}
To compute indicator values for a calendar quarter, one-year or other
result period when valid data may not exist for some portion of the
period, the following convention is used to determine whether a valid
indicator result can be obtained:
\begin{itemize}
\item To compute a quarterly value, valid data for contributing data elements must be present. (The only exception is the fuel reliability indicator; only one month of data is required to compute this indicator.)
\item To compute results for other periods, valid data must exist for
  at least half of the period. For example, for a one-year
  (four-quarter) value to be computed, at least two quarters of
  qualified data must exist; for a two-year (eight-quarter) value to
  be computed, at least four quarters of qualified data must exist.
\end{itemize}
WANO will provide updated results on the WANO website and add new
results for the latest quarter per unit, station and groups using
online reports as soon as possible after the quarterly data becomes
available. The reports will include:
\begin{itemize}
\item Standard reports, containing performance trend graphs of many WANO groups, updated upon the processing of the latest quarterly data using fixed criteria.
\item Custom reports in the format of summary tables containing lists of indicator results for individual units using criteria selected by the user.
\item Custom reports in the format of trend reports displaying performance trends and data of units or groups of units using criteria selected by the user.
\item Other reports, as developed to meet user needs or desires, such
  as “index” or “goal” reports to support the needs of members or
  regional centres.
\end{itemize}
WANO performance indicator programme data, including members, units, indicators results and selected plant information (such as reactor type, commercial operation date) will be available to members on the WANO website. This information is intended to facilitate verification of the reported information and to allow nuclear operating organisations to compare performance of similar plants, and initiate contact to emulate practices leading to improved performance. When comparing performance indicator results among plants, care must be taken to ensure comparison parameters are equivalent so that the comparison is accurate. The WANO Performance Indicator programme coordinators can assist WANO members in interpreting performance indicator information to help avoid errors of this type.
\subsection*{Use of WANO Performance Indicators}
WANO members are encouraged to share performance indicator
information, including the use of plant names, to allow comparison of
performance and emulation among plants.

Great care should be exercised in the use of performance indicators to
ensure that they are not used in a manner that could encourage plant
operating personnel to take non-conservative actions regarding plant
safety, in order to improve performance values or to achieve
performance goals that are based on the indicators. The following
principles should be applied when using the WANO performance
indicators:
\begin{itemize}
\item Performance indicators are most appropriately used by nuclear operating organisations for trending performance and, if needed, adjusting priorities and resources; the relative emphasis to be placed on a given indicator or set of indicators should be an operating line management prerogative.
\item Performance indicators should be used in conjunction with other assessment tools; they should not be used as the sole basis for decisions. Excessive focus on a narrow set of indicators or one indicator can be counter-productive to safety.
\item Detailed or process related indicators requiring a detailed knowledge of plant programmes should only be used by the plant staff; comparisons of such indicators can provide the plant staff with a useful perspective, but should not be used by other organisations to compare performance.
\item Performance indicators alone should not be used for ranking
  plants because they provide only a partial and historic perspective
  regarding safe and reliable plant operation.
\end{itemize}
It is important to respect the confidentiality of WANO performance indicators results for individual plants. Before quoting any performance indicator values outside of WANO, reference should be made to the current WANO Confidentiality Policy document.
\subsection*{Programme Administration}
LO defines the main steps of the PI programme and is responsible for
monitoring implementation status and programme effectiveness. Member
requested changes to the programme should contact their regional
centre. LO reviews inputs and recommendations from members or regional
centres for changes to the performance indicator programme. LO PI
System Administrator (SA) is responsible for the whole technical
support of members’ requests.

To facilitate implementation of the programme, and exchange of
experience, each regional centre and the London Office have identified
a point of contact knowledgeable of the region's performance indicator
programme.

Individual WANO members may elect to translate these guidelines into native languages or to supplement them in a common binder with member-specific reporting guidance.
\subsection*{Structure of this document}
This document is divided into several parts. The first part is a detailed description of the data elements, by category, that members submit periodically. The second part is a description of the indicators used by WANO and its members. The attachment has a description of Method 4 to calculate the WANO Index and description of long-term targets for individual units and industry for 2015 \& 2020. In addition, the examples of data elements and sample of indicator calculation are provided there.
