\subsection{Safety System Performance Indicator (SSPI) }

\subsubsection{Purpose}

The purpose of the safety system performance indicator is to monitor
the readiness of important safety systems to perform certain functions
in response to off-normal events or accidents. This indicator also
indirectly monitors the effectiveness of operation and maintenance
practices in managing the unavailability of safety system components.

The safety system performance indicator provides a simple indication
of safety system unavailability that shows good correlation with
results of system unavailability calculations using more precise
system modelling techniques (e.g. fault trees). A low value of the
safety system performance indicator indicates a greater margin of
safety for preventing reactor core damage and less chance of extended
plant shutdown due to failure of a safety system to function during an
operational event.

However, the objective should not be to attain a safety system performance indicator (unavailability) value that is near zero over a long term. Rather, the objective should be to attain a value that, while low, allows for maintenance activities to help maintain system reliability and availability consistent with safety analyses.
The safety system performance indicator is defined for the many
different types of nuclear reactors within the WANO membership. To
facilitate better understanding of the indicator and applicable system
scope for these different type reactors a separate section has been
developed for each reactor type.

Each section is written specifically for that reactor type and
reporting method. If a member desires to understand how a different
member is reporting or wishes to better understand that member's
indicator, it should consult the applicable section.

The various reactor type sections are:
\begin{itemize}
\item AGR
\item AP1000
\item BWR (including HWLWR)
\item GCR
\item PHWR
\item PWR
\item RBMK
\item VVER
\item FBR
\end{itemize}

\subsubsection{Safety System Performance Indicator (SSPI) AGR}
\paragraph{Scope}
The AGR safety systems monitored by this indicator are the following:
\begin{itemize}
\item Carbon dioxide supply system (SP1)
\item Post trip boiler feed systems (SP2)
\item Emergency AC power system (SP5)
\end{itemize}

These AGR systems were selected for the safety system performance
indicator based on their importance in preventing reactor core damage
or extended plant outage. Not every risk important system is
monitored. Rather, those that are generally important across the broad
nuclear industry are included within the scope of this indicator. They
include the principal systems needed for maintaining reactor coolant
inventory following a loss of primary coolant (CO2), for decay heat
removal following a reactor trip or loss of secondary coolant
(feedwater), and for providing emergency AC power following a loss of
plant offsite power.

Except as specifically stated in the definition and reporting
guidance, no attempt is made to monitor or give credit in the
indicator results for the presence of other systems at a given plant
that add diversity to the mitigation or prevention of accidents. For
example, no credit is given for additional power sources that add to
the reliability of the electrical grid supplying a plant because the
purpose of the indicator is to monitor the effectiveness of the
plant's response once the grid is lost.

\paragraph{Definitions}

The performance indicator is calculated separately for each of the
three systems for each reactor type. The safety system performance
indicator is defined for each safety system as the sum of the
unavailabilities, due to all causes, of the components (or emergency
generator trains) in the system during a time period, divided by the
number of trains in the system. The intent is to derive an
approximation of the average train unavailability due to component
unavailabilities. The emergency AC power system is treated at the
train level rather than at the component level, i.e. unavailability is
recorded only when the emergency generator is unavailable to produce
emergency AC power. The definition is further explained as follows:

\subparagraph {Component unavailability:} the fraction of time that a
component is unable to perform its intended function when it is
required to be available for service.

The component unavailability is the ratio of the hours the component
was unavailable (unavailable hours) to the hours the system was
required to be available for service.

\subparagraph {Component:} the equipment for which the unavailable
hours are recorded.

A component is included in the safety system performance indicator
monitored scope when unavailability of the component can degrade the
full capacity or redundancy of the system. For each safety system
additional guidance is provided below for determining the components
for which unavailable hours are monitored. Attachment gives an example
of calculation for this indicator.


\paragraph{Calculations}

The unit values for each safety system (except emergency AC power) are
determined for each reporting period as follows:

$$ \text{Value for a system} = \frac{\text{planned unavailable
      hours}+\text{unplanned unavailable hours}+\text{fault exposure
      unavailable hours}}{\text{hours system required} \cdot
      \text{number of trains}} $$

Because emergency generators (EGs) at multi-unit stations often serve
more than one unit, the indicator value for an emergency AC power
system is calculated for each period as a station7 value:

$$ \text{Value for a station} = \frac{\sum{\text{all unavailable hours for all
EGs}}}{\text{number of EGs} \cdot \text{hours in period}} $$

WANO Member Value: The indicator value (for each system) for a WANO
member is the median of the unit or station values of the indicator
for that system for the various plants represented by that member.

\paragraph{Reporting of Results}

The indicator results for each station will be displayed in periodic
plant-specific WANO formats.

In general, to allow more meaningful comparison of unit performance,
safety system performance for individual units will be presented for a
three-year period to minimise the effect of annual variations.

\paragraph{Scope of Components and Number of Trains}
\subparagraph{AGR Carbon Dioxide Delivery System}

This attachment provides additional guidance for the safety system
performance indicator for the AGR carbon dioxide supply system.

Scope of Components: This system consists essentially of those pumps,
vaporisers, vaporiser lines and injection routes which supply carbon
dioxide to the reactor vessel, including all associated valves and
pipework.

Carbon dioxide stocks are not included within the scope of this indicator.
Number of system trains: The number of trains equals the number of
vaporisers/injection routes/transfer lines dependent on system design
required by the safety case for normal operation.

\subparagraph{AGR Emergency Feed and Backup Cooling System}

This attachment provides additional guidance for the safety system performance indicator for the AGR emergency feed and back up cooling systems. These systems provide a means of supplying post trip reactor cooling.

\subparagraph{Emergency Feed System}

Scope of Components: This system consists essentially of those pumps
which supply water to the main boilers after a reactor trip
(emergency, start/standby boiler feed pumps and suction stage feed
pumps), and the associated water supply and steam venting
components. These pumps are electrically powered (with the exception
of two turbine driven emergency pumps at one station) and are supplied
from the site emergency electrical system.

Number of system trains8: The number of trains equals the number of
parallel pumps required by the safety case for normal operation. A
typical arrangement might be four electric emergency feed pumps per
site (turbine driven at some locations), any one of which is able to
feed the boilers in any shut down reactor and have complicated boiler
pressure control facilities. Various arrangements are possible,
depending on the station. The scope thus includes either the low
pressure venting system; or the boiler pressure control valves,
start-up vessel and dump condenser. Water supply is from the reserve
feed tanks on some stations, or the main feed and condensate system.

\subparagraph{Backup Cooling System}

Scope of components: The faults accommodated by this system are loss
of both main and emergency feed, but with the reactor
pressurised. Primary coolant flow is by means of natural circulation.

The pumps (usually three per station) for this system are usually
diesel driven but on one station are electric. They are housed in a
separate pump-house. Any one pump is able to deliver sufficient feed
to both shut down reactors simultaneously (all AGR stations have two
reactors). The pumps draw water from two dedicated, seismically
qualified water storage tanks, or a tank and a reservoir.

Fixed pipework connects the tanks to the pumps and the pumps to the
main feed header. All isolation and non-return valves in this pipework
are within the scope of the system to be monitored.

The pumps cannot supply feed until the boilers are partially
depressurised. Only that part of the boiler venting equipment
additional to that needed by the emergency feed system (that is, the
modified safety relief valves) is included within the inventory for
the backup feed system.

Number of system trains: The number of trains equals the number of
parallel pumps required by the safety case for normal operation.

\subsubsection{Safety System Performance Indicator (SSPI) AP1000 PWR}
\paragraph{Data Reporting}

The safety system performance indicator data will be reported on a
quarterly basis.

\paragraph{Scope}

The AP1000 is a pressurised water reactor (PWR). Its design however
departs significantly from current plants in its safety systems. PWR
design relies on pumps powered by alternating current (AC) and diesel
motors. Safety of the plant therefore depends on a reliable source of
that power.

The AP1000 PWR, in the event of a design-basis accident, is designed
to achieve and maintain safe shutdown condition without any operator
action and without the need for AC power or pumps. Instead of relying
on active components such as diesel generators and pumps,
the AP1000 relies on the natural forces of gravity, natural
circulation and compressed gases to keep the core and containment from
overheating.

The AP1000 safety systems monitored by SSPI are the following:
\begin{itemize}
\item High pressure safety injection (SP1) for reactor coolant inventory makeup:
\item Core Makeup Tanks (CMTs)
\item Automatic Depressurisation System (ADS) Stages 1, 2 and 3
\item Core decay heat removal (SP2):
\item Passive Residual Heat Removal Heat Exchanger (PRHR HX)
\item In-containment Refuelling Water Storage Tank (IRWST)
\item Emergency power system (SP5)
\end{itemize}

These systems were selected for the safety system performance
indicator based on their importance in preventing reactor core
damage. Not every risk important system is monitored. Rather, those
that are generally important across the broad nuclear industry are
included within the scope of this indicator. They include the
principal systems needed for maintaining reactor coolant inventory
following a loss of coolant, for decay heat removal following a
reactor trip and station blackout.

No attempt is made to monitor or give credit to the indicator results
for the presence of other systems at a given plant that add diversity
to the mitigation or prevention of accidents. For example, no credit
is given to non-safety related systems such as non-safety class power
sources that add to the reliability of the electrical power, because
the purpose of the indicator is to monitor the reliability of the
plant's safety related systems.

\paragraph{Definitions}

The performance indicator is calculated separately for each of the
three SSPIs; SP1, SP2 and SP5. The safety system performance indicator
is defined for each SSPI as the sum of the unavailabilities, due to
all causes, of the components in the system during a time period,
divided by the number of trains in the system. The intent is to derive
an approximation of the average train unavailability due to component
unavailabilities. The definition is further explained as follows:

\subparagraph{Component unavailability:} the fraction of time that a
component is unable to perform its intended function when it is
required to be available for service.

The component unavailability is the ratio of the hours the component
was unavailable (unavailable hours) to the hours the system was
required to be available for service.

\subparagraph{Component:} the equipment for which the unavailable hours are recorded.
A component is included in the safety system performance indicator
monitored scope when unavailability of the component can degrade the
full capacity or redundancy of the system. For each safety system
additional guidance is provided for determining the components for
which unavailable hours are monitored. Attachment 4 gives an example
of calculation.

\paragraph{Calculations}

The unit values for each safety system are determined for each
reporting period as follows:

$$ \text{Value for a system} = \frac{\text{planned unavailable
    hours}+\text{unplanned unavailable hours}+\text{fault exposure unavailable hours}}{\text{hours system required} \cdot \text{number of trains}} $$

WANO Member Value = the indicator value (for each system) for a WANO
member is the median of the unit values of the indicator for that
system for the various plants represented by that member.

\paragraph{Reporting of Results}

The indicator results for each station will be displayed in periodic
plant-specific WANO reports.

In general, to allow more meaningful comparison of unit performance,
it is more reasonable to present safety system performance for
individual units for a three-year period to minimise the effect of
annual variations.

\paragraph{Component Scope and Number of System Trains}

\subparagraph{AP1000 PWR High Pressure Safety Injection (SP1)}

This part provides additional guidance for the safety system
performance indicator for the AP1000 PWR passive safety injection
system. The components monitored for the system are those used
following a small cold leg loss of coolant accident (LOCA).

Scope of Passive SI system: Figure 1 shows a generic schematic of the
Passive SI system, indicating the components for which unavailable
hours are monitored. Plant-specific design differences may require
additional components to be included.

For the purpose of calculation for SSPI SP1, the unavailability of
Core Makeup Tanks (CMTs) and its associated valves and the stages 1 to
3 of Automatic Depressurisation System (ADS) is monitored.

For CMTs and ADS to be operable, the operability of valves given in
Figure 1 should be ensured. In addition, for CMTs, the requirements
for water volume, temperature limits and Boron concentration should be
fulfilled. The presence of non-condensable gases at high points of the
injection lines also contributes to the unavailability of the system.

The unavailable hours of any of the above system valves or out of
range of any water parameter of CMTs should be reported for
calculation of SP1.

Number of system trains:  The number of Passive SI system trains is defined by the number of Core Makeup Tanks (CMTs)

\subparagraph{AP1000 PWR Passive Core Cooling System (SP2)}

This part provides additional guidance for the safety system
performance indicator SP2 for the AP1000 PWR passive core cooling
system. The components monitored for the system are those used for
removal of decay heat following a loss of normal and startup feedwater
event.

Scope of passive core cooling system:  Figure 2 shows a generic
schematic, indicating the components for which unavailable hours are
monitored. For the purpose of calculation for SSPI SP2, the
unavailability of passive residual heat removal heat exchanger (PRHR
HX) and its associated valves, the in-containment refuelling water
storage tank (IRWST) and the IRWST gutter isolation valves are
monitored. Plant-specific design differences may require other
components to be included.

The non-safety class feedwater systems or the normal residual heat
removal system are not included in the scope of this performance
indicator.

For the passive core cooling system to be operable, the operability of
the passive residual heat removal heat exchanger (PRHR HX) and the
operability of valves given in Figure 2 should be ensured. In
addition, for the in-containment refuelling water storage tank
(IRWST), the requirements for water volume, temperature limits and
Boron concentration should be fulfilled. The presence of
non-condensable gases at high points of the core cooling lines and
PRHR HX also contributes to the unavailability of the system.

The passive core cooling system is assumed to be required for an
extended period of operation during which the water inventory of IRWST
is depleted. Therefore the IRWST gutter isolation valves are also
included in the monitored components for SP2.

The unavailable hours of any of the above system valves or out of
range of any water parameter of IRWST should be reported for
calculation of SP2.

Number of system trains:  The number of passive core cooling system
trains is defined by the number of passive residual heat removal heat
exchanger (PRHR HX). Therefore for AP1000 the number of trains for SP2
is one only.

\subparagraph{AP1000 PWR Emergency Power Systems (SP5)}

This part provides additional guidance for the safety system
performance indicator for the emergency power system.

Component scope for the emergency power system: The AP1000 is designed
to achieve and maintain safe shutdown condition without the need for
ac power or pumps. Instead of relying on active components such as
emergency diesel generators and pumps, the AP1000 relies on the
natural forces of gravity, natural circulation and compressed gases to
keep the core and containment from overheating.

The Class 1E DC and Uninterruptible Power Supply (UPS) System provides
electrical power for safety related and vital control instrumentation
loads, including monitoring and main control room emergency
lighting. It also provides power for safe shutdown when all the onsite
and offsite AC power sources are lost and cannot be recovered for 72
hours.

During normal operation, the DC load is powered from the battery
chargers with the batteries floating on the system. In case of loss of
normal power to the battery charger, the DC load is automatically
powered from the station batteries. Each battery bank provides power
to an inverter, which in turn powers an AC instrumentation and control
bus. The AC instrumentation and control bus loads are connected to
inverters according to the battery bank type, 24 hour or 72 hour.

For the purpose of calculation of SP5, the unavailability hours of
Class 1E DC power system batteries and chargers and Class 1E AC
instrumentation and control buses inverters and UPS shall be monitored
and reported.

For the purpose of calculation of SP5, the unavailability hours of the
following Class 1E and UPS system components shall be monitored and
reported: batteries, battery chargers, inverters, fused transfer
switches, DC switchboards, and AC instrumentation and control buses.

Number of system trains:  The number of emergency power system trains
is equal to the number of Class 1E DC electrical power subsystems


\subsubsection{Safety System Performance Indicator (SSPI) BWR (includes heavy – water
moderated, boiler light water cooled reactors)}

\paragraph{Scope}

The BWR safety systems monitored by this indicator are the following:
\begin{itemize}
\item High pressure injection/heat removal systems (e.g. high pressure coolant injection, high pressure core spray, or feedwater coolant injection; and reactor core isolation cooling or isolation condenser systems). For calculating the safety system performance indicator, data for these systems are combined to represent a single multi-train system that functions at high pressure to maintain reactor coolant inventory and to remove decay heat following a loss of main feedwater event, or to mitigate a small break LOCA. (SP1)
\item Residual heat removal system9 (SP2)
\item Emergency AC power system (SP5)
\end{itemize}

These systems were selected for the safety system performance
indicator based on their importance in preventing reactor core damage
or extended plant outage. Not every risk important system is
monitored. Rather, those that are generally important across the broad
nuclear industry are included within the scope of this indicator. They
include the principal systems needed for maintaining reactor coolant
inventory following a loss of coolant, for decay heat removal
following a reactor trip or loss of main feedwater, and for providing
emergency AC power following a loss of plant offsite power

Note: This scope applies to heavy water moderated, boiling light water
cooled reactor (HWLWR) systems.

Except as specifically stated in the definition and reporting
guidance, no attempt is made to monitor or give credit in the
indicator results for the presence of other systems at a given plant
that add diversity to the mitigation or prevention of accidents. For
example, no credit is given for additional power sources that add to
the reliability of the electrical grid supplying a plant because the
purpose of the indicator is to monitor the effectiveness of the
plant's response once the grid is lost.

\paragraph{Definitions}

The performance indicator is calculated separately for each of the
three systems for each reactor type. The safety system performance
indicator is defined for each safety system as the sum of the
unavailabilities, due to all causes, of the components (or emergency
generator trains) in the system during a time period, divided by the
number of trains in the system. The intent is to derive an
approximation of the average train unavailability due to component
unavailabilities. The emergency AC power system is treated at the
train level rather at the component level, i.e. unavailability is
recorded only when the emergency generator is unavailable to produce
emergency AC power. The definition is further explained as follows:

\subparagraph{Component unavailability:} the fraction of time that a
component is unable to perform its intended function when it is
required to be available for service

The component unavailability is the ratio of the hours the component
was unavailable (unavailable hours) to the hours the system was
required to be available for service.

\subparagraph{Component:} the equipment for which the unavailable
hours are recorded A component is included in the safety system
performance indicator monitored scope when unavailability of the
component can degrade the full capacity or redundancy of the
system. For each safety system additional guidance is provided below
for determining the components for which unavailable hours are
monitored. Attachment 4 gives an example of calculation.

Data for BWR high pressure injection/heat removal systems are
collected separately for several systems that perform similar safety
functions. However, for calculating the safety system performance
indicator, the data for these systems are combined to represent a
single multi-train system that functions at high pressure to maintain
reactor coolant inventory and to remove decay heat following a loss of
main feedwater event, or to mitigate a small break LOCA.

\paragraph{Calculations}

The unit values for each safety system (except emergency AC power) are
determined for each reporting period as follows:

$$ \text{Value for a system} =
\frac{\text{planned unavailable hours}+\text{unplanned unavailable hours}+\text{fault
exposure unavailable hours}}{\text{hours system required} \cdot \text{number of
trains}} $$

Because emergency generators (EGs) at multi-unit stations often serve
more than one unit, the indicator value for an emergency AC power
system is calculated for each period as a station value as follows:

$$ \text{Value for a station} = \frac{\sum{\text{all unavailable hours for all
EGs}}}{\text{number of EGs} \cdot \text{hours in period}} $$

WANO Member Value = The indicator value (for each system) for a WANO
member is the median of the unit or station values of the indicator
for that system for the various plants represented by that member.

\paragraph{Reporting of Results}

The indicator results for each station will be displayed in periodic
plant-specific WANO reports in a manner that shows both the overall
indicator value and that portion of the value caused by planned
activities.

In general, to allow more meaningful comparison of unit performance,
safety system performance for individual units will be presented for a
three-year period to minimise the effect of annual variations.

\paragraph{BWR High Pressure Injection/Heat Removal Systems (High
  Pressure Coolant Injection, High Pressure Core Spray, Feedwater
  Coolant Injection, Reactor Core Isolation Cooling, and Isolation
  Condenser Systems)}

This part provides additional guidance for the safety system
performance indicator for five systems: high pressure coolant
injection (HPCI), high pressure core spray (HPCS), feedwater coolant
injection (FWCI), reactor core isolation cooling (RCIC), and the
isolation condenser. Plants should monitor either HPCI, HPCS, or FWCI,
and either RCIC or the isolation condenser, depending on which systems
are installed. However, for calculating the safety system performance
indicator, the data for these two systems are combined to represent a
single, multi-train system that functions at high pressure to maintain
reactor coolant inventory and to remove decay heat for an extended
period following a loss of main feedwater event, or to mitigate a
small break LOCA.

Scope of systems:  Figures \ref{fig:3.1} show generic
schematics for the HPCI, HPCS, RCIC, isolation condenser, and FWCI
systems respectively. These schematics indicate the components for
which unavailable hours are monitored. Plant-specific design
differences may require other components to be included.

These systems are assumed to be required for an extended period of
time during which the initial supply of water from the condensate
storage tank is depleted and recirculation of water from the
suppression pool (or water from another source, for isolation
condensers) is required. Therefore, components in the flow paths from
both of these water sources are included.

The HPCS system typically includes a ‘water leg’ pump to prevent water
hammer in the HPCS piping to the reactor vessel. The ‘water leg’ pump
and valves in the ‘water leg’ pump flow path are not included in the
scope of the HPCS system for the performance indicator.

The HPCS system has a dedicated emergency diesel generator. This diesel generator is not included in the scope of the HPCS system, but is included in the emergency AC power system scope.
HPCI and RCIC turbines and associated valves and piping for steam
supply and exhaust are in the scope of these systems.

\subparagraph{Number of system trains:}  The HPCI system is considered
a single-train system. The booster pump and other small pumps shown in
Figure \ref{} are not used to determine the number of trains; however,
the unavailable hours for these pumps are included in the safety
system performance indicator.

The RCIC system is also considered a single-train system. The
condensate and vacuum pumps shown in Figure \ref{} are not used in
determining the number of trains; however, the unavailable hours for
these pumps are included in the safety system performance indicator.

The isolation condenser system is considered to be a single-train
system for units with one isolation condenser and a two-train system
for units with two 100 per cent capacity or four 50 per cent capacity
isolation condensers.

For the feedwater coolant injection system, the number of trains is
determined by the number of main feedwater pumps that can be used in
this operating mode at one time. Condensate and feedwater booster
pumps are not used to determine the number of trains.

The number of trains in the combined system is the sum of the number
of trains in the individual systems. Thus, a plant with a single train
HPCI system and a single-train RCIC system would be defined as having
a two-train system for determining the value of the safety system
performance indicator.


\paragraph{BWR Residual Heat Removal System}

This part provides additional guidance for the safety system
performance indicator for the BWR residual heat removal (RHR) system
and for other systems used to remove heat to outside containment under
low pressure conditions at several early BWRs with unique designs.

\subparagraph{RHR Systems:}  The RHR system has several modes of
operation. The two modes considered to be of greatest importance for
preventing core damage (based on probabilistic risk studies) are the
suppression pool cooling mode and the shutdown cooling mode. The
suppression pool cooling mode is used whenever the suppression pool
(or torus) water temperature exceeds a high temperature set-point
(e.g. following most relief valve openings). The shutdown cooling mode
is used following any transient requiring long term heat removal from
the reactor vessel.

Component scope for the residual heat removal system:  Figure 4 shows
a generic schematic indicating the components in the RHR system for
which unavailable hours are monitored. Plant-specific design
differences may require other components to be included.

Although the low pressure coolant injection (LPCI) mode of RHR
operation is not specifically included in the scope of this indicator,
the components used for the LPCI mode at most BWRs are monitored as
part of the suppression pool cooling and shutdown cooling modes.

Components on the secondary side of the RHR heat exchangers are not
included.

\subparagraph{Other Systems:}  For some BWRs with unique designs,
other systems are used to remove heat to outside the containment under
low pressure conditions. Depending on the particular design, one or
more of the following systems may be used:  shutdown cooling,
containment spray, low pressure coolant injection (torus cooling
mode), or the post incident system.

Component scope for unique BWR heat removal systems:  All components
required for each safety system to perform its heat removal function
should be included in the scope.

The number of trains required to be in service is defined in the data
element part.

\paragraph{Emergency AC Power Systems}

Diesel generators dedicated to providing emergency power for BWR high
pressure core spray (HPCS) pumps are included as part of the emergency
AC power system10.


\subsubsection{Safety System Performance Indicator (SSPI) GCR}
\paragraph{Data Reporting}

The safety system performance indicator data will be reported on a
quarterly basis. For further information see the data reporting
instructions.

\paragraph{Scope}

The GCR (gas cooled reactors) safety systems monitored by this
indicator are the following:

\subparagraph{Magnox Reactors}
\begin{itemize}
\item emergency feed system (SP1)
\item tertiary feed system (SP2)
\item emergency AC or DC power system (SP5)
\end{itemize}

These GCR systems were selected for the safety system performance
indicator based on their importance in preventing reactor core damage
or extended plant outage. Not every risk important system is
monitored. Rather, those that are generally important across the broad
nuclear industry are included within the scope of this indicator. They
include the principal systems needed for maintaining reactor coolant
inventory following a loss of coolant, for decay heat removal
following a reactor trip or loss of main feedwater, and for providing
emergency AC power following a loss of plant offsite power. (Gas
cooled reactors have an additional decay heat removal system instead
of the coolant inventory maintenance system).

Except as specifically stated in the definition and reporting
guidance, no attempt is made to monitor or give credit in the
indicator results for the presence of other systems at a given plant
that add diversity to the mitigation or prevention of accidents. For
example, no credit is given for additional power sources that add to
the reliability of the electrical grid supplying a plant because the
purpose of the indicator is to monitor the effectiveness of the
plant's response once the grid is lost.

\paragraph{Definitions}

The performance indicator is calculated separately for each of the
three systems for each reactor type. The safety system performance
indicator is defined for each safety system as the sum of the
unavailabilities, due to all causes, of the components (or emergency
generator trains) in the system during a time period, divided by the
number of trains in the system. The intent is to derive an
approximation of the average train unavailability due to component
unavailabilities. The emergency AC power system is treated at the
train level rather at the component level, i.e. unavailability is
recorded only when the emergency generator is unavailable to produce
emergency AC power. The definition is further explained as follows:

\subparagraph{Component unavailability:} the fraction of time that a
component is unable to perform its intended function when it is
required to be available for service

The component unavailability is the ratio of the hours the component
was unavailable (unavailable hours) to the hours the system was
required to be available for service.

\subparagraph{Component:} the equipment for which the unavailable
hours are recorded

A component is included in the safety system performance indicator
monitored scope when unavailability of the component can degrade the
full capacity or redundancy of the system. For each safety system
additional guidance is provided below for determining the components
for which unavailable hours are monitored. Attachment 4 shows example
calculation for this indicator.

\paragraph{Calculations}

The unit values for each safety system (except emergency AC power) are
determined for each reporting period as follows:

$$ \text{Value for a system} =
\frac{\text{planned unavailable hours}+\text{unplanned unavailable hours}+\text{fault
exposure unavailable hours}}{\text{hours system required} \cdot \text{number of
trains}} $$

Because emergency generators (EGs) at multi-unit stations often serve
more than one unit, the indicator value for an emergency AC power
system is calculated for each period as a station11 value as follows:

$$ \text{Value for a station} = \frac{\sum{\text{all unavailable hours for all
EGs}}}{\text{number of EGs} \cdot \text{hours in period}} $$

WANO Member Value = The indicator value (for each system) for a WANO
member is the median of the unit or station values of the indicator
for that system for the various plants represented by that member.

\paragraph{Reporting of Results}

The indicator results for each station will be displayed in periodic plant-specific WANO formats.
In general, to allow more meaningful comparison of unit performance,
safety system performance for individual units will be presented for a
three-year period to minimise the effect of annual variations.

\paragraph{Scope of Components and Number of Trains}

\subparagraph{Emergency Feed System}

This part provides additional guidance for the safety system
performance indicator for the Magnox Emergency Feed System (EFS).

Scope of Components:  This system consists essentially of those pumps
which supply water to the main boilers after a reactor trip, and the
associated water supply and steam venting components. These pumps are
electrically powered (with the exception of two turbine driven
emergency pumps at one station) and are supplied from the site
emergency electrical system.

The main and the start/standby feed pumps are used for normal
operations (that is, their primary purpose is in the electrical
production process, even though there are times when they may be used
for shut down heat removal) rather than part of a safety system and
hence are not included in the scope of the EFS.

Number of system trains12:  The number of trains equals the number of
parallel pumps. The typical arrangement at the Magnox stations is four
electric emergency feed pumps per site, any one of which is able to
feed the boilers in any shut down reactor. Water supply is from the
reserve feed tanks. Steam from the boilers goes to the flash vessel
and then to the station dump condenser or it is vented to
atmosphere. All these components and the associated pipework and
valves are considered to be part of the system.

\subparagraph{Tertiary Feed}

This part provides additional guidance for the safety system
performance indicator for the Magnox Tertiary Feed System. This system
provides a means of supplying post trip reactor cooling in the event
of certain hazards disabling both the main and the emergency feed
systems. The hazards considered vary from site to site but the
principal ones are:
\begin{itemize}
\item loss of all onsite electric power
\item turbine hall major steam release
\item turbine hall major fire
\item turbine hall major flood
\item turbine disintegration
\item seismic disturbance
\end{itemize}

Some station designs are less vulnerable than others to some items on
the list on account of plant disposition or segregation. For instance,
sites with only four emergency generators have diesel driven pumps for
this system; electric driven systems have seven or eight emergency
generators.

\subparagraph{Scope of Tertiary Feed System}

The faults accommodated by this system are loss of both main and
emergency feed, with the reactor either pressurised or depressurising
slowly. Primary coolant flow by means of natural circulation is
sufficient. Tertiary feed flow cannot begin until the boilers have
depressurised.  This depressurisation requires no special venting
arrangements, but happens naturally as a consequence of the fault
transients for which the Tertiary Feed System provides protection.

Most stations have several low pressure diesel driven pumps (typically
three per reactor) housed in a dedicated building(s) and augmented by
a mobile diesel driven pump. Only one pump is needed to accommodate
the various faults. Each pump has its own independent fuel supply and
unitised battery start arrangements. The feed water source consists of
a towns water reservoir or seismically qualified tank, or both. The
pipework to the pumps and from the pumps to the emergency boiler feed
main may be rigid or flexible (that is, fire hose type). Various
isolation and non-return valves are included, which permit coupling
from either source to any pump to any reactor. Until the point of
final connection to the emergency feed main, all parts of the tertiary
feed system are diverse from the other feed systems.

A variation on the above arrangements occurs at one station. A robust
reservoir-fed town's main (fire main type system) supplies low
pressure water to the reactors. This is augmented by two mobile diesel
pumps which obtain their supplies from an onsite river.

\subparagraph{Number of system trains:}  The number of trains is equal to the number of parallel pumps.
*See reporting guidelines to determine if some pumps qualify as
installed spares.

This part provides additional guidance for the safety system
performance indicator for the emergency AC power system.


\subsubsection{Safety System Performance Indicator (SSPI) PHWR}

Heavy water moderated, boiling water cooled reactors (including HWLWR)
are in BWR section.

\paragraph{Data Reporting}

The safety system performance indicator data will be reported on a
quarterly basis. For further information see the data reporting
instructions.

\paragraph{Scope}

The safety systems monitored by this indicator are the following:

\subparagraph{PHWRs}

Although the PHWR safety philosophy considers other special safety
systems to be paramount to public safety, the following PHWR safety
and safety-related systems were chosen to be monitored in order to
maintain a consistent international application of the safety system
performance indicators.
\begin{itemize}
\item high pressure emergency coolant injection system (SP1)
\item auxiliary boiler feedwater system (SP2)
\item emergency AC power (SP5)
\end{itemize}

These systems were selected for the safety system performance
indicator based on their importance in preventing reactor core damage
or extended plant outage. Not every risk important system is
monitored. Rather, those that are generally important across the broad
nuclear industry are included within the scope of this indicator. They
include the principal systems needed for maintaining reactor coolant
inventory following a loss of coolant, for decay heat removal
following a reactor trip or loss of main feedwater, and for providing
emergency AC power following a loss of plant offsite power.

Except as specifically stated in the definition and reporting
guidance, no attempt is made to monitor or give credit in the
indicator results for the presence of other systems at a given plant
that add diversity to the mitigation or prevention of accidents. For
example, no credit is given for additional power sources that add to
the reliability of the electrical grid supplying a plant because the
purpose of the indicator is to monitor the effectiveness of the
plant's response once the grid is lost.

\paragraph{Definitions}

The performance indicator is calculated separately for each of the
three systems for each reactor type. The safety system performance
indicator is defined for each safety system as the sum of the
unavailabilities, due to all causes, of the components (or emergency
generator trains) in the system during a time period, divided by the
number of trains in the system. The intent is to derive an
approximation of the average train unavailability due to component
unavailabilities. The emergency AC power system is treated at the
train level rather at the component level, i.e. unavailability is
recorded only when the emergency generator is unavailable to produce
emergency AC power. The definition is further explained as follows:

\subparagraph{Component unavailability:} the fraction of time that a component is
unable to perform its intended function when it is required to be
available for service

The component unavailability is the ratio of the hours the component
was unavailable (unavailable hours) to the hours the system was
required to be available for service.

\subparagraph{Component:} the equipment for which the unavailable
hours are recorded.

A component is included in the safety system performance indicator
monitored scope when unavailability of the component can degrade the
full capacity or redundancy of the system. For each safety system
additional guidance is provided for determining the components for
which unavailable hours are monitored. Attachment 4 gives specific
guidance for the calculation.

\paragraph{Calculations}

The unit values for each safety system (except emergency AC power) are
determined for each reporting period as follows:

$$ \text{Value for a system} =
\frac{\text{planned unavailable hours}+\text{unplanned unavailable hours}+\text{fault
exposure unavailable hours}}{\text{hours system required} \cdot \text{number of
trains}} $$

Because emergency generators (EGs) at multi-unit stations often serve
more than one unit, the indicator value for an emergency AC power
system is calculated for each period as a station13 value as follows:

$$ \text{Value for a station} = \frac{\sum{\text{all unavailable hours for all
EGs}}}{\text{number of EGs} \cdot \text{hours in period}} $$

WANO Member Value = the indicator value (for each system) for a WANO member is the median of the unit or station values of the indicator for that system for the various plants represented by that member.
Reporting of Results
The indicator results for each station will be displayed in periodic plant-specific WANO reports in a manner that shows the overall indicator value.
In general, to allow more meaningful comparison of unit performance, safety system performance for individual units will be presented for a three-year period to minimise the effect of annual variations.
Scope of Components and Number of Trains
PHWR High Pressure Emergency Coolant Injection, Auxiliary Boiler Feedwater and Emergency AC Power Systems
This part provides additional guidance for safety systems to be monitored at pressurised heavy water reactor (PHWR) plants.  Although the PHWR safety philosophy considered other special safety systems to be paramount to public safety, the high pressure emergency coolant injection (HPECI), auxiliary boiler feedwater (ABF), and emergency AC power (EACP) systems are monitored to maintain consistent international application of this indicator.
Scope of Components: The following section provides information regarding the scope of components to be monitored for each PHWR safety system listed above.
The components monitored for the HPECI system are those used following a loss-of-coolant accident (LOCA). The HPECI is assumed to be required for an extended time following initial injection. Therefore, all components in the long-term recovery portions of the system are also monitored for this indicator.  Figure 5.1 shows a generic schematic of the HPECI system, indicating the components for which unavailable hours are monitored.
The components monitored for the ABF system include those used to remove heat from the steam generator following a reactor trip and loss of main feedwater capability due to loss of the main feedwater pumps. Figure5.2 shows a generic schematic of the ABF system, indicating the components for which unavailable hours are monitored. Some PHWRs may possess other emergency feedwater systems to respond to events other than loss of main feedwater pumps. However, to allow consistent comparison of results with other PHWRs, only those components in the ABF system are monitored for this indicator.
The components monitored for the EACP system include those used following a loss of offsite power. Figure 5.3 shows the scope of the components monitored for a typical emergency generator. Components or sub-systems not included in this scope (e.g. fuel supply system) are considered to be support systems that, if unavailable, can cause the emergency generator to be unavailable. To maintain consistency between PHWRs with different emergency AC power system designs, additional standby generator sets installed for use following seismic events are not included in the EACP system scope for purposes of this indicator.
Number of system trains: The following section provides information for determining the number of system trains for each PHWR safety system.
The number of HPECI trains is determined by the number of parallel prime movers (e.g. pumps or accumulators), independent of flow capacity, installed to perform the high pressure injection function. Prime movers used in the low pressure injection phase and recovery phase are not counted when determining the number of trains. However, these prime movers and their associated valves are included in the system scope and are monitored for unavailable hours. The number of ABF trains is determined by the number of prime movers, independent of flow capacity, installed to remove decay heat from steam generators following a reactor trip or loss of main feedwater pumps.
The number of EACP trains determined by the number of emergency generator sets (as described in the section on system scope) installed to power shutdown loads in the event of a loss of offsite power.






Safety System Performance Indicator (SSPI) PWR
Data Reporting
The safety system performance indicator data will be reported on a quarterly basis.
Scope
The PWR safety systems monitored by this indicator are the following:
high pressure safety injection system (SP1)
auxiliary feedwater system (SP2)
emergency AC power system (SP5)
These systems were selected for the safety system performance indicator based on their importance in preventing reactor core damage or extended plant outage. Not every risk important system is monitored. Rather, those that are generally important across the broad nuclear industry are included within the scope of this indicator. They include the principal systems needed for maintaining reactor coolant inventory following a loss of coolant, for decay heat removal following a reactor trip or loss of main feedwater, and for providing emergency AC power following a loss of plant offsite power.
Except as specifically stated in the definition and reporting guidance, no attempt is made to monitor or give credit in the indicator results for the presence of other systems at a given plant that add diversity to the mitigation or prevention of accidents. For example, no credit is given for additional power sources that add to the reliability of the electrical grid supplying a plant because the purpose of the indicator is to monitor the effectiveness of the plant's response once the grid is lost.
Definitions
The performance indicator is calculated separately for each of the three systems for each reactor type. The safety system performance indicator is defined for each safety system as the sum of the unavailabilities, due to all causes, of the components (or emergency generator trains) in the system during a time period, divided by the number of trains in the system. The intent is to derive an approximation of the average train unavailability due to component unavailabilities. The emergency AC power system is treated at the train level rather at the component level, i.e. unavailability is recorded only when the emergency generator is unavailable to produce emergency AC power. The definition is further explained as follows:
Component unavailability: the fraction of time that a component is unable to perform its intended function when it is required to be available for service
The component unavailability is the ratio of the hours the component was unavailable (unavailable hours) to the hours the system was required to be available for service.
Component: the equipment for which the unavailable hours are recorded
A component is included in the safety system performance indicator monitored scope when unavailability of the component can degrade the full capacity or redundancy of the system. For each safety system additional guidance is provided for determining the components for which unavailable hours are monitored. Attachment 4 gives example of calculation.
Calculations
The unit values for each safety system (except emergency AC power) are determined for each reporting period as follows:
$$ Value for a system =
((planned unavailable hours)+(unplanned unavailable hours)+(fault
exposure unavailable hours))/((hours system required) \cdot (number of
trains)) $$
Because emergency generators (EGs) at multi-unit stations often serve more than one unit, the indicator value for an emergency AC power system is calculated for each period as a station14 value as follows:
$$ Value for a station = ((sum of all unavailable hours for all
EGs))/((number of EGs) \cdot (hours in period)) $$
Each element has been described in the Data Element part of this manual.
WANO Member Value = the indicator value (for each system) for a WANO member is the median of the unit or station values of the indicator for that system for the various plants represented by that member.
Reporting of Results
The indicator results for each station will be displayed in periodic plant-specific WANO reports in a manner that shows both the overall indicator value and that portion of the value caused by planned activities.
In general, to allow more meaningful comparison of unit performance, safety system performance for individual units will be presented for a three-year period to minimise the effect of annual variations.
Scope of Components and Number of Trains
PWR High Pressure Safety Injection System
This part provides additional guidance for the safety system performance indicator for the PWR high pressure safety injection (HPSI) system. The components monitored for the HPSI system are those used following a small cold leg loss of coolant accident (LOCA).
Scope of HPSI system:  There are design differences among HPSI systems that affect the scope of the components to be included for the HPSI system function. Figure 1.1 shows a generic schematic of the HPSI system, indicating the components for which unavailable hours are monitored. Plant-specific design differences may require additional components to be included.
The HPSI system is assumed to be required for an extended period of time during which the initial supply of water from the refuelling water storage tank is depleted and recirculation of water from the containment sump is required. Therefore, components in the flow paths from both of these water sources are included.
Some PWR plants have centrifugal high head pumps in the chemical and volume control system (CVCS) that are considered part of the high pressure safety injection system. For these plants, the CVCS high head pumps and associated HPSI flow path should be included in the scope of components for the HPSI safety system performance indicator. Figure 1.2 shows a schematic of the components to be included for this HPSI configuration. Positive displacement pumps, because of these limited flow capacity, should not be included in the HPSI system scope.
Some plants have one HPSI pump continuously operating as a charging pump. Such a continuously operating HPSI pump is included in the scope of the HPSI system components to be monitored. Unavailable hours would be counted only for the repair time of a continuously operating train, because the time of any failure would be accurately known.
At some plants, recirculation of water from the containment sump requires that the high pressure injection pump take suction via the low pressure injection/residual heat removal pumps. For these plants, the low pressure injection/residual heat removal pumps and the associated valves in the flow path from the sump to the HPSI pumps are included in the scope of components for the HPSI system. Figure 1.3 shows a schematic of the components to be included for this HPSI configuration.
Some plants have a HPSI train that requires a manual start. For the purpose of this performance indicator, this train is included in the scope of components for which unavailable hours are monitored.
Number of system trains:  The number of HPSI system trains is defined by the number of parallel high pressure pumps used to perform the HPSI function. For example, a system with two dedicated HPSI pumps and two high head CVCS pumps that start on a safety injection signal is considered a four-train HPSI system. When high pressure pumps are required to take suction from low pressure system pumps, the low pressure pumps should not be considered as additional trains. Installed spare pumps should not be used to determine the number of system trains.







PWR Auxiliary Feedwater System
This part provides additional guidance for the safety system performance indicator for the PWR auxiliary feedwater (AFW) system. The components monitored for the AFW system are those used following a reactor trip or loss of main feedwater.
Scope of AFW system:  Figure 2 shows a generic schematic, indicating the components for which unavailable hours are monitored. Plant-specific design differences may require other components to be included.
Some plants have a startup feedwater pump that requires a manual start. Startup feedwater pumps are not included in the scope of the AFW system for this performance indicator.
The AFW system is assumed to be required for an extended period of operation during which the initial supply of water from the condensate storage tank is depleted and water from an alternative water source (e.g. the service water system) is required. Therefore components in the flow paths from both of these water sources are included; however, the alternative water source (e.g. service water system) is not included.
Number of system trains:  The number of system trains is determined by the number of parallel pumps in the AFW system, not by the number of injection lines. For example, systems with three AFW pumps are defined as three-train systems, whether they feed two, three, or four injection lines, and regardless of the flow capacity of the pumps.



Emergency AC Power Systems 15
Description/Scope
Additional guidance is provided for reporting performance of the emergency AC power system. The emergency AC power system is typically comprised of two or more independent emergency generators that provide AC power to class 1E buses following a loss of offsite power.
The function monitored for the indicators is:
The ability of the emergency generators to provide AC power to the class 1E buses upon a loss of offsite power.
Most (i.e. diesel-driven) emergency generator trains include dedicated subsystems such as air start, lube oil, fuel oil, cooling water, etc. Support systems can include service water, DC power, and room cooling. Generally, unavailable hours are counted if a failure or unavailability of a dedicated subsystem or a support subsystem prevents the emergency generator from performing its function. Some examples are discussed in the clarifying notes for this part.
The electrical circuit breaker(s) that connect(s) an emergency generator to the class 1E buses that are normally served by that emergency generator are considered to be part of the emergency generator train.
Emergency generators that are not safety grade, or that serve a backup role only (e.g. an alternate AC power source), are not required to be included in the performance reporting.
Train Determination
The number of emergency AC power system trains is equal to the number of emergency generators that are installed at the unit to power safe-shutdown loads in the event of a loss of offsite power, including the diesel generator dedicated to the High Pressure Core Spray (HPCS) system.

Safety System Performance Indicator (SSPI) RBMK
Data Reporting
The safety system performance indicator data will be reported on a quarterly basis. For further information see the data reporting instructions.
Scope
The RBMK safety systems monitored by this indicator are the following:
first emergency heat removal system (SP1)
second emergency heat removal system (SP2)
emergency AC power (SP5)
These RBMK systems were selected for the safety system performance indicator based on their importance in preventing reactor core damage or extended plant outage. Not every risk important system is monitored. Rather, those that are generally important across the broad nuclear industry are included within the scope of this indicator. They include the principal systems needed for maintaining reactor coolant inventory following a loss of coolant, for decay heat removal following a reactor trip or loss of main feedwater, and for providing emergency AC power following a loss of plant offsite power.
Except as specifically stated in the definition and reporting guidance, no attempt is made to monitor or give credit in the indicator results for the presence of other systems at a given plant that add diversity to the mitigation or prevention of accidents. For example, no credit is given for additional power sources that add to the reliability of the electrical grid supplying a plant because the purpose of the indicator is to monitor the effectiveness of the plant's response once the grid is lost.
Definitions
The performance indicator is calculated separately for each of the three systems. The safety system performance indicator is defined for each safety system as the sum of the unavailabilities, due to all causes, of the components (or emergency generator trains) in the system during a time period, divided by the number of trains in the system. The intent is to derive an approximation of the average train unavailability due to component unavailabilities. The emergency AC power system is treated at the train level rather at the component level, i.e. unavailability is recorded only when the emergency generator is unavailable to produce emergency AC power. The definition is further explained as follows:
Component unavailability: the fraction of time that a component is unable to perform its intended function when it is required to be available for service
The component unavailability is the ratio of the hours the component was unavailable (unavailable hours) to the hours the system was required to be available for service.
Component: the equipment for which the unavailable hours are recorded
A component is included in the safety system performance indicator monitored scope when unavailability of the component can degrade the full capacity or redundancy of the system. For each safety system additional guidance is provided for determining the components for which unavailable hours are monitored. Attachment 4 gives an example of calculation.

Calculations
The unit values for each safety system (except emergency AC power) are determined for each reporting period as follows:
$$ Value for a system =
((planned unavailable hours)+(unplanned unavailable hours)+(fault
exposure unavailable hours))/((hours system required) \cdot (number of
trains)) $$
Because emergency generators (EGs) at multi-unit stations often serve more than one unit, the indicator value for an emergency AC power system is calculated for each period as a station16 value as follows:
$$ Value for a station = ((sum of all unavailable hours for all
EGs))/((number of EGs) \cdot (hours in period)) $$
WANO Member Value = The indicator value (for each system) for a WANO member is the median of the unit or station values of the indicator for that system for the various plants represented by that member.
Reporting of Results
The indicator results for each station will be displayed in periodic plant-specific WANO formats.
In general, to allow more meaningful comparison of unit performance, safety system performance for individual units will be presented for a three-year period to minimise the effect of annual variations.
Component Scope and Number of System Trains
RBMK Emergency Heat Removal Systems
This part provides additional guidance for the safety system performance indicator for RBMK emergency heat removal system (EHRS).
Component Scope and Number of System Trains:
RBMK
For RBMK reactors the EHRS consist of two trains, each consisting of the following subsystems (fig 1), that performs sequentially, depending on pressure in channels of the reactor:
first emergency heat removal subsystem having passive acting part with pressurised hydro-accumulators and active part with one pump in each train,
second emergency heat removal subsystem having six pumps connected to accidental half core (item 2 in fig 1) and three pumps connected to non-accidental half core (item 1 in fig 1).



Figure 1 Emergency Heat Removal System RBMK Reactors (Example of Reporting Scope)

Safety System Performance Indicator (SSPI) VVER
Data Reporting
The safety system performance indicator data will be reported on a quarterly basis. For further information see the data reporting instructions.
Scope
The VVER safety systems monitored by this indicator are the following:
high pressure safety injection system (SP1)
auxiliary feedwater system (including both auxiliary and emergency feedwater pumps) (SP2)
emergency AC power system (SP5)
These VVER systems were selected for the safety system performance indicator based on their importance in preventing reactor core damage or extended plant outage. Not every risk important system is monitored. Rather, those that are generally important across the broad nuclear industry are included within the scope of this indicator. They include the principal systems needed for maintaining reactor coolant inventory following a loss of coolant, for decay heat removal following a reactor trip or loss of main feedwater, and for providing emergency AC power following a loss of plant offsite power.
Except as specifically stated in the definition and reporting guidance, no attempt is made to monitor or give credit in the indicator results for the presence of other systems at a given plant that add diversity to the mitigation or prevention of accidents. For example, no credit is given for additional power sources that add to the reliability of the electrical grid supplying a plant because the purpose of the indicator is to monitor the effectiveness of the plant's response once the grid is lost.
Definitions
The performance indicator is calculated separately for each of the three systems. The safety system performance indicator is defined for each safety system as the sum of the unavailabilities, due to all causes, of the components (or emergency generator trains) in the system during a time period, divided by the number of trains in the system. The intent is to derive an approximation of the average train unavailability due to component unavailabilities. The emergency AC power system is treated at the train level rather at the component level, i.e. unavailability is recorded only when the emergency generator is unavailable to produce emergency AC power. The definition is further explained as follows:
Component unavailability: the fraction of time that a component is unable to perform its intended function when it is required to be available for service
The component unavailability is the ratio of the hours the component was unavailable (unavailable hours) to the hours the system was required to be available for service.
Component: the equipment for which the unavailable hours are recorded
A component is included in the safety system performance indicator monitored scope when unavailability of the component can degrade the full capacity or redundancy of the system. For each safety system additional guidance is provided for determining the components for which unavailable hours are monitored. Attachment 4 gives an example of calculation.

Calculations
The unit values for each safety system (except emergency AC power) are determined for each reporting period as follows:
$$ Value for a system =
((planned unavailable hours)+(unplanned unavailable hours)+(fault
exposure unavailable hours))/((hours system required) \cdot (number of
trains)) $$
Because emergency generators (EGs) at multi-unit stations often serve more than one unit, the indicator value for an emergency AC power system is calculated for each period as a station17 value as follows
$$ Value for a station = ((sum of all unavailable hours for all
EGs))/((number of EGs) \cdot (hours in period)) $$
WANO Member Value = The indicator value (for each system) for a WANO member is the median of the unit or station values of the indicator for that system for the various plants represented by that member.
Reporting of Results
The indicator results for each station will be displayed in periodic plant-specific WANO formats.
In general, to allow more meaningful comparison of unit performance, safety system performance for individual units will be presented for a three-year period to minimise the effect of annual variations.
Component Scope and Number of System Trains
VVER High Pressure Safety Injection System
This part provides additional guidance for the safety system performance indicator for the VVER high pressure safety injection (HPSI) system. The components monitored for the HPSI system are those used following a small cold leg loss of coolant accident (LOCA) or uncontrolled cooldown of reactor coolant system.
Scope of HPSI system: Figures 1.1 and 1.2 provide schematics of usual HPSI system designs of VVER plants showing the components to be monitored for the indicator purpose. There may be design differences among HPSI systems, however, that affect the scope of the components to be included for the HPSI system function. Plant-specific design differences may therefore require additional components to be monitored.
The HPSI system is assumed to be required for an extended period of time during which the initial supply of water from the refuelling water storage tank is depleted and recirculation of water from the containment sump is required. Therefore, components in the flow paths from both of these water sources are included.
Number of system trains:  The number of HPSI system trains is defined by the number of parallel high pressure pumps used to perform the HPSI function. For example, a system with three dedicated make-up HPSI pumps and three dedicated emergency boration HPSI pumps that start on a safety injection signal (VVER-1000) is considered a six-train HPSI system. Installed spare pumps should not be used to determine the number of system trains.


Figure 1.1 Schematic of VVER-440/B213 HPSI


Figure 1.2 Schematic of VVER-1000 HPSI

VVER Auxiliary Feedwater System
This part provides additional guidance for the safety system performance indicator for the VVER auxiliary feedwater (AFW) system used following a reactor trip or loss of main feedwater.
The components monitored for the AFW system include both auxiliary and emergency feedwater pumps and valves
Scope of AFW system: Figures 2.1 and 2.2 provide schematics of usual AFW system designs of VVER plants showing the components to be monitored for this indicator purpose. However, there may be design differences of AFW systems at various plants, which may require additional components to be included.
The AFW system is assumed to be required for an extended period of operation during which the initial supply of water from the feedwater tank (dearator) is depleted and water from an alternative water source (e.g. the condensate storage system) is required. Therefore components in the flow paths from both of these water sources are included; however, the alternative water source (e.g. condensate storage system) is not included.
Number of system trains:  The number of system trains is determined by the number of parallel pumps in the AFW system, not by the number of injection lines. For example, AFW system with two auxiliary feedwater pumps and two emergency feedwater pumps is defined as four-train system, whether it feeds two, three, or four injection lines, and regardless of the flow capacity of the pumps.


Figure 2.1 Schematic of VVER-440 AFW System


Figure 2.2 Schematic of VVER-1000 AFW System

Safety System Performance Indicator (SSPI) FBR
Scope
The Fast Breeder Reactor (FBR) safety systems monitored by this indicator are as follows:
No safety injection system exists for maintaining reactor inventory (This SSPI is denoted by SP1)
Residual heat removal system (This SSPI is denoted by SP2)
Emergency AC power system (This SSPI is denoted by SP5)
These systems were selected for the safety system performance indicator based on their importance in preventing reactor core damage or extended plant outage.  Not every safety system is monitored.  Rather, those that are generally important across the broad nuclear industry are included within the scope of this indicator. They include the principal systems needed for decay heat removal following a reactor trip and for providing emergency AC power following a loss of plant off-site power. FBR has a decay heat removal system with air cooler heat sink air cooler either by natural circulation or forced circulation.
As to Emergency AC power system, more detail description and definition are given in “Safety System Performance   Emergency AC Power System (SP5)“ together with this guidance description.
Except as specifically stated in the definition and reporting guidance, no attempt is made to monitor or give credit in the indicator results for the presence of other systems at a given plant that add diversity to the mitigation or prevention of accidents.  For example, no credit is given for additional power sources that add to the reliability of the electrical grid supplying a plant because the purpose of the indicator is to monitor the effectiveness of the plant's response once the grid is lost.
Definitions
The performance indicator is calculated separately for each of the two systems (SP2 \& SP5) for FBR.  The safety system performance indicator is defined for each safety system as the sum of the unavailabilities, due to all causes, of the components (or emergency generator trains) in the system during a time period, divided by the number of trains in the system.
These parts are the following:
FBR safety system for reactor inventory
FBR residual heat removal system
Emergency AC power systems (See the same sheet of the light water reactors (BWR, PWR)).
Calculation
The unit values for each safety system (except emergency AC power) are determined for each reporting period as follows:
unit value for a system =  ((planned unavail hours)+(unplanned unavail hours)+(fault exposure unavail hours))/( (hours system required-default value) x (number of trains))

Because emergency generators (EGs) at multi-unit stations often serve more than one unit, the indicator value for an emergency AC power system is calculated for each period as a station18 value as follows:
station value =  ((sum of all unavailable hours for all EGs))/((average number of EGs present during the period) x (hours in period))
WANO Member Value
The indicator value (for each system) for a WANO member is the median of the unit or station values of the indicator for that system for the various plants represented by that member.
Reporting of results
The indicator results for each station will be displayed in periodic plant-specific WANO reports in a manner that shows both the overall indicator value and that portion of the value caused by planned activities.
In general, to allow more meaningful comparison of unit performance, safety system performance for individual units will be presented for a three-year period to minimise the effect of annual variations.
An example of the data collected for this performance indicator and a sample calculation for a FBR system are provided in the Attachment 4.
FBR Reactor Inventory
The figures show the system overviews of Monju which is the typical FBR plant developed in Japan. The coolant in the reactor is safely kept by dual vessel structure (Reactor and Guard Vessel) even if the reactor boundary failure event occurred. That is why FBR does not need any safety injection systems and SP1 is not defined for the plant.

Figure 1 MONJU Plant Overview \& FBR unique PIs


Figure 2 MONJU reactor core inventory
FBR Residual Heat Removal (RHR) Systems
This part provides additional guidance for safety systems to be monitored at FBR plants reactor cooling function. In general, FBR needs residual heat transfer systems to remove decay heat from the core safely after the rector shut down. This safety system should be monitored as SP2 in SSPI for safe operation of FBR plant (see the figure below).
Scope of Components: The following section provides information regarding the scope of components to be monitored for each safety systems.
The components monitored for the RHR system include those in the primary and secondary heat transfer system and auxiliary cooling system connected to the secondary system. Figure 2 shows the schematic diagram of RHR system in Monju plant. Safety operation performance of these system components are monitored by SP2.
Number of system trains: The number of system trains is determined by the number of parallel pumps in the both primary and secondary sodium system. For example, systems with three primary and secondary sodium systems are three-train systems, regardless of the flow capacity of the pumps
 Unit-based SP5 calculation
Each unit in a multi-unit site reports the unavailable hours of all AC emergency power systems for each unit and not any more the total of the site.
All information related to “Emergency AC Power” in DES will be collected at a unit based in a similar way as data elements for “High Pressure Safety Injection” and “Auxiliary Feedwater”.
DES will be modified by WANO LO SA during period which is necessary to collect all AC Emergency Power System (EPS) design information from RC and end-users.

Currently, the SP5 is calculated based on the unavailability of all AC emergency power systems on site, without differentiation for multi-unit sites. The modification allows the calculation of SP5 for each unit on the same site (data elements collected for each unit).
$$ Value for a unit (if appropriate) = ((sum of all unavailable hours
for unit's EGs))/((number of unit's EGs) \cdot (hours in period)) $$
$$ Value for a station (if appropriate) = ((sum of all unavailable
hours for all EGs))/((number of EGs) \cdot (hours in period)) $$
The PI REPORT system will provide unit-based and station-based information separately. The long-term target values for both indicators are the same and defined into Section Three, Attachment \ref{Att}, page \pageref{Att}.

Figure 3 MONJU RHR Systems and relevant components
