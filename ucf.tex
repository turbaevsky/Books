\subsection{Unit Capability Factor (UCF)}
\subsubsection{Purpose}
The purpose of this indicator is to monitor progress in attaining high
unit and industry energy production reliability. This indicator
reflects effectiveness of plant programs and practices in maximising
available electrical generation, and provides an overall indication of
how well plants are operated and maintained.

\subsubsection{Definition}
Unit capability factor is defined as the ratio of the available energy
generation over a given time period to the reference energy generation
over the same time period, expressed as a percentage. Both of these
energy generation terms are determined relative to reference ambient
conditions.

Available energy generation is the energy that could have been
produced under reference ambient conditions considering only
limitations within control of plant management, i.e. plant equipment
and personnel performance, and work control.

Reference energy generation is the energy that could be produced if
the unit were operated continuously at full power under reference
ambient conditions.

Reference ambient conditions are environmental conditions
representative of the annual mean (or typical) ambient conditions for
the unit.

\subsubsection{Calculations}

The unit capability factor is determined for each period as shown
below:

$$ \text{Value for a unit} = \frac{REG-PEL-FEL-OEL}{REG}\cdot 100 $$
Where

REG  =	reference energy generation for the period

PEL   =	total planned energy losses for the period

FEL   =	unplanned forced energy losses for the period

OEL  =	unplanned outage energy losses for the period

$$\text{Value for the industry} = \text{median of the unit values}$$

Three-year periods minimise the impact of annual variations due to refuelling and planned maintenance outages. An example of the data collected for this performance indicator and a sample calculation are provided in the Attachment \ref{UCF}.
