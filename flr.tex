\subsection{Forced Loss Rate (FLR)}

\subsubsection{Purpose}

The purpose of this indicator is to monitor industry progress in
minimising outage time and power reductions that result from unplanned
equipment failures, human errors, or other conditions during the
operating period (excluding planned outages and their possible
unplanned extensions). This indicator reflects the effectiveness of
plant programs and practices in maintaining systems available for safe
electrical generation when the plant is expected to be at the grid
dispatcher’s disposal.

\subsubsection{Definition}

The forced loss rate (FLR) is defined as the ratio of all unplanned
forced energy losses during a given period of time to the reference
energy generation minus energy generation losses corresponding to
planned outages and any unplanned outage extensions of planned
outages, during the same period, expressed as a percentage. 

Unplanned energy losses are either unplanned forced energy losses
(unplanned energy generation losses not resulting from an outage
extension) or unplanned outage extension of planned outage energy
losses. 

Planned energy losses are those corresponding to outages or power
reductions which were planned and scheduled at least 4 weeks in
advance (see clarifying notes for exceptions).

\subsubsection{Calculations}

The forced loss rate is calculated for a period as shown below.
$$ \text{Value for a unit (\%)} = \frac{FEL}{REG-PEL-OEL} \cdot 100 $$		
Where	

FEL =	unplanned forced energy losses

REG =	reference energy generation 

PEL =	planned energy losses 

 OEL =	 unplanned outage extension energy losses

$$ \text{Value for the industry} = \text{median of the unit values} $$