\subsection{Unplanned Capability Loss Factor (UCLF)}

\subsubsection{Purpose}

The purpose of this indicator is to monitor industry progress in
minimising outage time and power reductions that result from unplanned
equipment failures or other conditions. This indicator reflects the
effectiveness of plant programs and practices in maintaining systems
available for safe electrical generation.

\subsubsection{Definition}
Unplanned capability loss factor is defined as the ratio of the
unplanned energy losses during a given period of time, to the
reference energy generation, expressed as a percentage.

Unplanned energy loss is energy that was not produced during the
period because of unplanned shutdowns, outage extensions, or unplanned
load reductions due to causes under plant management control. Causes
of energy losses are considered to be unplanned if they are not
scheduled at least four weeks in advance. Causes considered to be
under plant management control are further defined in the clarifying
notes.

Reference energy generation is the energy that could be produced if
the unit were operated continuously at full power under reference
ambient conditions throughout the period. Reference ambient conditions
are environmental conditions representative of the annual mean (or
typical) ambient conditions for the unit.

\subsubsection{Calculation}
The unplanned capability loss factor is determined for each period as shown below:
$$ Value for a unit  =  \frac{FEL+OEL}{REG} \cdot 100 $$
Where

FEL  = unplanned forced energy losses for the period

OEL = total unplanned outage energy losses for the period

REG = reference energy generation for the period

$$ Value for the industry  = median of the unit values $$

Three-year periods minimise the impact of annual variations due to
refuelling and planned maintenance outages. An example of the data
collected for this performance indicator and a sample calculation are
provided in the Attachment \ref{UCLF}.

As a point of interest, the sum of unit capability factor, unplanned capability loss factor, and planned capability loss factor equals 100\% over a specific time period. Planned capability loss factor can be calculated from this relationship.
