\section{Data Elements}
\subsection{Generation}
\subsubsection{Reference Unit Power}

The Reference Unit Power (RUP) is the maximum power capability of the
unit under reference ambient conditions. If a maximum power capability
has been determined by formal test, the reference unit power is
determined by correcting test results to reference ambient
conditions. If a formal test has not been performed, the RUP should be
based on design values, adjusted for reference ambient conditions.

RUP is expected to remain constant unless design changes that affect
the capacity are made to the unit or as a result of a new permanent
authorisation. Intentional plugging of steam generator tubes is
considered a plant design modification. The reference unit power
should be re-evaluated and revised if necessary to account for the
reduced maximum capability of the unit due to the intentionally
plugged steam generator.

The reference ambient conditions are environmental conditions
representative of the annual mean (or typical) conditions for a
unit. It is expected that historical heat sink temperatures will be
used to determine the reference ambient conditions. The same reference
ambient conditions will generally apply for the life of the
unit. Periodic review of these reference conditions is not required.

RUP value should also include the electrical equivalent of the portion
of energy delivered in the form of steam and/or heat that might have
been used for non-electrical applications. However, this applies only
to units in which heat production may reduce the unit electrical power
below its maximum value.

Either net or gross energy may be used; however, consistency must be
maintained for all energy terms.

\subsubsection{Reference Energy Generation (REG)}
The REG is determined by multiplying the RUP by the period hours. REG
is expressed in units of megawatt-hours (electric).

\paragraph{Energy Losses}
Energy losses are the energy that was not produced during the period,
expressed in units of megawatt-hours (electric).

Energy losses are considered to be unplanned if they are not scheduled
at least four weeks in advance.

Energy losses due to reduced power are always related to the RUP. The
value of power losses to be used in computing energy losses due to a
particular event are the losses that would have occurred if the unit
were operating at the RUP level at the time of the event.

The power losses relative to the RUP may be determined by one of the
following techniques:
\begin{itemize}
\item Subtracting the actual power level during the event from the power level immediately prior to the event when the power was at the RUP level
\item Computing the power level reduction that would have occurred with the unit at the RUP level
\item Using historical data from similar events occurring at the RUP
  level
\end{itemize}

For example, if a unit experiences a 10 MW power loss due to an
equipment problem while operating at 75\% of the RUP, and it is
determined from calculations or from similar events that have occurred
at the RUP that the same equipment problem would have resulted in a 20
MW power loss at the RUP level, then 20 MW should be used when
computing the energy loss.

Energy losses for unrelated but concurrent equipment failures under
plant management control that result in generation losses are
determined as if the energy losses occurred separately. (Total energy
losses reported, however, cannot exceed reference energy generation
with priority given to unplanned energy losses during operation).

If energy losses during an event occur due to a combination of causes
under management control and causes outside of management control, the
portion of the total generation losses that are unplanned and are
under management control are identified and included when computing
the reportable losses.

Energy losses related to load reduction preceding a shutdown and load
increases following the shutdown should be categorised as planned or
unplanned depending on whether the shutdown is planned or
unplanned. For example, energy losses while entering and recovering
from an unplanned outage are considered to be unplanned forced energy
losses. If an outage extension occurs at the end of a planned outage,
the energy loss during ‘recovery’ from the outage will still be
considered as a planned energy loss because the shutdown was
originally caused by a planned outage. Energy losses due to required
tests following refuelling are considered planned energy losses.

A unit in reserve shutdown will be considered as available if it can
be restarted within the normal time required for unit start-up. If
work on plant equipment is undertaken that would prevent a restart,
the energy that potentially could have been produced while the plant
was unavailable should be computed and reported as planned or
unplanned energy loss, even if the plant was not actually required to
start-up during the period.

\subsubsection{Planned Energy Loss (PEL)}
Planned energy loss (PEL) is the energy that was not produced during the period because of planned shutdowns or load reductions due to causes under plant management control. Energy losses are considered to be planned if they are scheduled at least four weeks in advance.
$$PEL = \sum{(PPL \cdot HRP)}$$
Where:

PPL (planned power loss) is the power decrease in megawatts due to a
planned event

HRP (hours of reduced power) is the hours operated at reduced power or
shutdown due to the planned event

For the energy loss data determination, ‘planned’ includes
specifically being ‘scheduled’.

Note:	The total planned energy loss for the period is the sum of the
losses from all planned events.

Planned energy losses (those scheduled at least four weeks in advance)
caused by the following conditions should be included when computing
the unit capability factor because they are considered to be under the
control of plant management:
\begin{itemize}
\item refuelling or planned maintenance outages
\item planned outages or load reductions for testing, repair, or other
  plant equipment or personnel-related causes
\end{itemize}

Energy losses due to tests may be considered as planned if they are
identified at least four weeks in advance and are part of a regular
programme, even if the precise time of the test is not decided four
weeks in advance.

In general, changes in an outage or load reduction start date must be
announced at least four weeks in advance to be considered as planned.

However, if a change in the start date is requested by the grid
dispatcher less than four weeks in advance, the outage or load
reduction is considered to be planned, assuming this decision is due
to all of the following reasons or circumstances:
\begin{itemize}
\item The unit is considered as able to run with no unplanned losses during the four-week period prior to the initial planned outage start date.
\item Any forced or unplanned outage occurring during this four-week
  period (or before the new start date) shall not become the reason
  for putting forward the planned outage.
\end{itemize}

\subsubsection{Unplanned Forced Energy Loss (FEL)}

Unplanned forced energy loss (FEL) is energy that was not produced
because of unplanned shutdowns or unplanned load reductions due to
causes under plant management control when the unit is considered to
be at the disposal of the grid dispatcher.

Unplanned energy losses resulting from the following conditions are
considered to be under the control of plant management:
\begin{itemize}
\item Unplanned maintenance outages, excluding extensions of planned outages. (Extensions of planned outages are always reported as unplanned outage extension energy losses if the extension was planned and scheduled less than four weeks before the extension began or the originally scheduled work was not being completed. If the extension was planned and scheduled at least four weeks before the extension began, the losses may be included as planned energy losses.)
\item Unplanned outages or load reductions for unplanned testing, repair, or other plant equipment or personnel-related causes.
\item Unplanned outages or load reductions that are caused or
  prolonged by regulatory actions taken as a result of plant equipment
  or personnel performance, or regulatory actions applied. This also
  applies on a general basis to all similar plants, excluding those
  associated with extensions of planned outages.
\end{itemize}

If a unit begins an outage or load reduction before the scheduled
start date, the energy loss from the beginning of the outage or load
reduction to the scheduled start date is an unplanned forced energy
loss.

In case of generation losses due to equipment degradation (heat
exchangers, valve steam leaks, etc.) the energy losses would be
reported as unplanned forced losses until the degradation is repaired
or the unit enters an outage.

If equipment problem cannot be fixed immediately and requires a
temporary design change (for example, isolation of the failed
component) which reduces the power for months until the final repair
is done, the related power loss (for the whole period of reduced
power) should be classified as unplanned until all corrective
maintenance required for operation at full power operation is
completed.

Unplanned energy losses due to the following causes are not reported because these losses are not considered to be under the control of the plant management:
\begin{itemize}
\item grid instability or failure
\item lack of demand (reserve shutdown, economic shutdown, or load following)
\item environmental limitations (e.g. low cooling pond level, water intake restrictions, earthquakes or deluges that could not be prevented by operator action)
\item fuel
\item seasonal variations in gross dependable capacity due to cooling water temperature variations
fuel conservation directed by regulatory authority
\item labour strike
\end{itemize}

If a labour strike occurs during plant operation, any outage or load
reduction, generation losses due to the strike are not reportable. If
the strike occurs during a planned and scheduled outage, the original
scheduled end date of the planned outage is considered to have been
extended by the duration of the strike. The revised scheduled end date
is used to determine planned losses and outage extension losses once
the strike is over. If the strike occurs during an unplanned/forced
outage, reportable unplanned losses resume after the strike is over.

\subsubsection{Unplanned Outage Extension Energy Loss (OEL)}

Unplanned outage extension energy loss (OEL) is energy that was not
produced because of an extension of a planned outage or load reduction
beyond the scheduled planned end date due to originally scheduled work
not being completed, or because newly scheduled work was added
(planned and scheduled) to the outage less than 4 weeks before the
scheduled end of the planned outage.

Planned energy losses are those corresponding to outages or power
reductions which were planned and scheduled at least 4 weeks in
advance.

For events involving unplanned outages and start up following these
outages, reference unit power is used as the basis for computing power
losses.

If energy losses during an event occur due to a combination of causes
under management control and causes outside of management control, the
portion of the total generation losses that are unplanned and are
under management control are identified and used when computing the
energy losses.

If a unit begins an outage or load reduction before the scheduled start date, the energy loss from the beginning of the outage or load reduction to the scheduled start date is an unplanned forced energy loss.
If an outage extends beyond the scheduled start up date, either to
complete originally scheduled work or to complete corrective
maintenance work on equipment required for startup, all energy loss
associated with the outage extension is considered an unplanned outage
extension loss, not forced.

In general, changes in an outage or load reduction start date must be
announced at least four weeks in advance to be considered as planned.

However, if a change in the start date is requested by the grid
dispatcher less than four weeks in advance, the outage or load
reduction is considered to be planned, assuming this decision is due
to all of the following reasons or circumstances:
\begin{itemize}
\item The unit is considered as able to run with no unplanned losses during the four week period prior to the initial planned outage start date.
\item Any forced or unplanned outage occurring during this four week
  period (or before the new start date) shall not become the reason
  for putting forward the planned outage
\end{itemize}

Forced losses are reported until the cause of the losses is fixed or
until the start date of a planned outage scheduled before the cause of
the loss.

Energy losses that occur while entering and recovering from an
unplanned outage are considered to be unplanned forced energy
losses. If an outage extension occurs at the end of a planned outage,
the energy loss during ‘recovery’ from the outage will still be
considered as a planned energy loss because the shutdown was
originally caused by a planned outage. Energy losses due to required
tests following refuelling are considered planned energy losses.

The scheduled start and end dates of planned outages and load
reductions are those dates negotiated with and agreed to by the
network and/or grid dispatcher at least 4 weeks in advance. These
dates may differ from dates used at the unit for directing the outage.

\subsubsection{Energy Losses due to Grid Instability or Loss of Grid
  (GLR)}

Grid instability energy loss is energy that was not produced because
of instability or abnormal voltage or frequency conditions on the
electrical grid to which the unit is connected. The associated
shutdowns or load reductions are due to causes NOT under plant
management control.

Loss of grid energy loss is energy that was not produced because of
the loss of the electrical grid to which the unit is connected. The
loss of grid is due to causes NOT under plant management control.

Grid instability or failure energy losses, due to causes not
considered to be under the control of the plant management, include,
but are not limited to:
\begin{itemize}
\item grid frequency fluctuations
\item grid equipment failures
\item grid operational errors
\item sudden loss of load
\end{itemize}

Energy losses due to the following causes are NOT included/reported
because these losses are not related to grid instability or failure:
\begin{itemize}
\item lack of demand (reserve shutdown, economic shutdown, or load following)
\item environmental limitations (for example, low cooling pond level or water intake restrictions that could not be prevented by operator action)
\item fuel coastdowns
\item seasonal variations in gross dependable capacity due to cooling water temperature variations.
\item Industrial action (labour strikes) – Outages or load reductions
  caused by labour strikes that occur while the unit is operating are
  normally not included as unplanned energy losses because these
  energy losses are not under the direct control of plant
  management. However, if during the strike the unit becomes incapable
  of starting or operating because of equipment failures, maintenance,
  or other unplanned activities, then the energy losses during the
  time the unit is inoperable are included.
\end{itemize}

If energy losses during an event occur due to a combination of causes
under management control and causes outside of management control, the
applicable portions of the total generation losses are identified and
reported in accordance with the guidance.

Clarifying notes specifying “outage” also apply to a “load reduction”
that is not a complete unit shutdown.

\paragraph{Examples:}

The following examples are provided to help members determine
reportability of the grid-related generation losses.
\begin{itemize}
\item The electrical grid served by a nuclear unit has experienced voltage oscillations, causing the unit to reduce power. The grid oscillations were due to large amounts of smoke and heat near the power lines from a ground grass fire. The generation losses are reported as grid-related generation losses.
\item  A nuclear unit experienced an automatic reactor scram as a result of an equipment failure on the grid. The equipment was not under plant management control. The generation losses resulting from the scram are reported as grid-related generation losses. (The automatic scram is also reported, even though the cause was not under management control, but it meets other scram reportability guidance.)
\item  A unit operating at 50\% power due to planned plant maintenance. The unit then needed to quickly reduce power due to grid operator errors. Reported grid-related generation losses include those had the unit been operating at 100\% power – not just those below 50\% power. The planned generation losses are also reported separately as planned losses.
\item  A lightning strike on the grid (that the unit protection is not expected to absorb) causes the unit to trip (automatic scram). The generation losses are reported as grid-related losses.
\item  A unit was shut down in anticipation of severe weather. The weather resulted in extensive damage to grid equipment (“the grid”). The unit was available to operate following the storm, but the condition of the grid forced the unit to remain shut down. All generation losses due to the grid unavailability are reported as grid-related generation losses. The generation losses in anticipation of the storm are not reportable (in any category) as they were caused by the weather which is not under management control. There were no generation losses due to grid-related causes prior to the storm damage.
\item  Inadequate cooling fan maintenance on a main transformer causes a loss of output to the grid. The resulting generation losses are not reported as grid-related losses, but as unplanned generation losses, since the cause of the loss was under management control (maintenance).
\item  A unit was directed to reduced power due to lack of demand on
  the grid due to low energy consumption during a holiday period. The
  generation losses are not reportable. The cause is not related to
  grid issues or problems, but a lack of demand; the cause of the
  generation loss was not under management control.
\end{itemize}

\subsubsection{Number of Unplanned Automatic Scrams (AS)}

The number of automatic scrams1 while critical is defined as follows:
\begin{itemize}
\item Scram means that the shutdown was not an anticipated part of a planned test.
\item Automatic scram means the automatic shutdown of the reactor by a rapid insertion of negative reactivity (e.g. by control rods, liquid injection shutdown system, etc.) that is caused by actuation of the reactor protection system. The scram signal may have resulted from exceeding a setpoint or may have been spurious.
\item Automatic means that the initial signal that caused actuation of the reactor protection system logic was provided from one of the sensors monitoring plant parameters and conditions, rather than the manual scram switches or, in certain cases described in the clarifying notes, manual turbine trip switches (or pushbuttons) provided in the main control room.
\item Critical means that during the steady-state condition of the reactor prior to the scram, the effective multiplication factor (keff) was essentially equal to one.
\item Scrams that are planned to occur as part of a test (e.g. a reactor protection system actuation test) or scrams that are part of a normal operation or evolution and are covered by controlled procedures are not included.
\item Reactor protection system actuation signals that occur while all control rods are inserted are not counted, because no control rod movement occurred as a result of the signals.
\item During a startup, shutdown or changing power condition, the reactivity transients may cause the reactor to go subcritical or super-critical for a short period of time. However, the plant is considered critical for purposes of this indicator if the reactor was critical prior to the reactivity transient. It may be assumed to return to a critical condition after the transient is completed (e.g. a plant is considered to remain critical after initial criticality is declared on a reactor startup, and to be critical until taken permanently subcritical on a reactor shutdown).
\item Each scram caused by intentional manual tripping of the turbine, should be analysed to determine those which clearly involve a conscious decision by the operator to manually trip the turbine, to protect important equipment or to minimise the effects of a transient. Scrams that involve such a decision are considered manual scrams and are not counted as automatic scrams.
\item All unplanned reactor scrams must be reported, even if they
  occurred after the unit was disconnected from grid (when the reactor
  remained at power and critical).
\end{itemize}

\subsubsection{Number of Unplanned Manual Scrams (MS)}

The number of unplanned manual scrams while critical is defined as
follows:
\begin{itemize}
\item Unplanned means that the scram was not an anticipated part of a planned test.
\item Manual scram means the manual shutdown of the reactor by a rapid insertion of negative reactivity (e.g. by control rods or liquid injection shutdown system, etc.). This is caused by actuation of the reactor protection system, manual scram switches, manual turbine trip switches (or pushbuttons).
\item Manual means the initial signal that caused actuation of the reactor protection system logic, was provided from manual scram switches or, in certain cases described in the clarifying notes, manual turbine trip switches (or pushbuttons).
\item Critical means that during the steady-state condition of the reactor prior to the scram, the effective multiplication factor (keff) was essentially equal to one.
\item All unplanned reactor scrams must be reported, even if they
  occurred after the unit was disconnected from grid (when the reactor
  remained at power and critical) and even if they occurred due to
  reasons considered not being under management control.
\end{itemize}

Data for new units are included in the calculation of industry values
beginning the first full calendar quarter of operation, following the
commercial operation date. However, in order to be included in the
industry values, the unit must have at least 1,000 critical hours per
year.
\begin{itemize}
\item Scrams that are planned to occur as part of a test (e.g. a reactor protection system actuation test) or that are part of a normal operation or evolution and are covered by controlled procedures, are not included.
\item Reactor protection system actuation signals that occur while all control rods are inserted, are not counted, because no control rod movement occurred as a result of the signals.
\item During a startup, shutdown or changing power condition, the
  reactivity transients may cause the reactor to go subcritical or
  super-critical for a short period of time. However, the plant is
  considered to be critical for purposes of this indicator if the
  reactor was critical prior to the reactivity transient. It may be
  assumed to return to a critical condition after the transient is
  completed (e.g. a plant is considered to remain critical after
  initial criticality is declared on a reactor startup, and to be
  critical until taken permanently subcritical on a reactor shutdown).
\end{itemize}

\paragraph{Total Hours Critical in Period}
It is the number of hours when the reactor is critical.

Critical means that during the steady-state condition of the reactor
prior to the scram, the effective multiplication factor (keff) was
essentially equal to one.

In order to be included in the industry values, the unit must have at
least 1,000 critical hours per year. Requiring this minimum number of
critical hours reduces the effects of plants that are shut down for
long periods of time and whose limited data may not be statistically
valid.

\paragraph{Example Determination of Data Elements}

The following examples are provided to illustrate when reactor
protection system actuations are counted or are not counted as
unplanned scrams:

Reactor Protection System Actuations:
\begin{itemize}
\item  While shutting down the reactor, sufficient control rods had been inserted to make the reactor subcritical. A spurious scram signal then caused the remaining control rods to insert into the core. (This scram is not counted for the performance indicator because the reactor was not critical.)
\item  An automatic scram occurred while conducting a special test on the turbine. The plant procedure used for this test indicated that a scram would occur while performing the test. (This scram is not counted for the performance indicator because the scram is part of a planned operation and is covered by plant procedures.)
\item While conducting a routine surveillance test of the reactor protection system at 100\% power, an automatic scram occurred when a spurious signal was received on one protection system channel while another channel was being tested. (This scram is counted for the AS.)
\item While at full power, a main feedwater pump tripped. Operators attempted to restart the pump and to reduce reactor power, but actions to maintain steam generator (PWR) or reactor (BWR) levels were unsuccessful. Operators then initiated a manual scram before the setpoint for an automatic scram was reached. (This scram counts for the MS but not for the AS because the scram did not result from an automatic actuation of the reactor protection system.)
\item While at 75\% power, operators tripped the main turbine to prevent overspeed caused by a malfunction in the turbine control system. The turbine trip caused an automatic scram. (This scram counts for as a manual scram because the scram was caused by operators manually tripping the turbine to prevent equipment damage.)
Total number of scrams reported based on the example above would be one automatic scram (AS = 1) and two manual scrams (MS = 2).
\end{itemize}

\subsection{Equipment Performance}
\subsubsection{Safety Systems Performance Indicator (SSPI)}
\paragraph{Data Reporting}
The safety system performance indicator data elements will be reported
on a quarterly basis.

The intent is to derive an approximation of the average train
unavailability due to component unavailabilities. The emergency
alternating current (AC) power system is treated at the train level
rather at the component level, i.e. unavailability is recorded only
when the emergency generator is unavailable to produce emergency AC
power. The definition is further explained as follows:
\begin{itemize}
\item Component unavailability: the fraction of time that a component
  is unable to perform its intended function when it is required to be
  available for service.

The component unavailability is the ratio of the hours the component
was unavailable (unavailable hours) to the hours the system was
required to be available for service.

\item Component: the equipment for which the unavailable hours are
  recorded.

A component is included in the safety system performance indicator
monitored scope when unavailability of the component can degrade the
full capacity or redundancy of the system. A section in the indicator
part for each safety system provides additional guidance for
determining the components for which unavailable hours are
monitored. Attachment gives calculation specific guidance.
\end{itemize}

Data for BWR high pressure injection/heat removal systems are
collected separately for several systems that perform similar safety
functions. However, for calculating the safety system performance
indicator, the data for these systems are combined to represent a
single multi-train system that functions at high pressure to maintain
reactor coolant inventory and to remove decay heat following a loss of
main feedwater event, or to mitigate a small break Loss Of Coolant
Accident (LOCA).

\paragraph{Data Elements}

The following data elements are required to determine each unit's
values for this indicator for each safety system:

\subparagraph{Component/train unavailable hours}

The sum of the hours that each monitored component/train in the system
was unavailable to perform its intended function, due to all causes,
while the system was required to be available for service.

Unavailable hours for various fluid systems (e.g. BWR high pressure
injection/heat removal and residual heat removal systems) are
monitored by many stations at the component level because of the
difficulty in associating all components/trains in some of these
systems with distinct trains.

The emergency AC power system is always monitored at the train level
rather than the component level. That is, unavailable hours are
recorded only when the emergency generator train is unavailable to
deliver emergency AC power. The failure of one of two redundant
emergency generator support subsystems, for example, would not count
toward emergency generator unavailability as long as the emergency
generator train was still available. See the relevant section in the
indicator calculation part for further explanation. Additional details
regarding data reporting for the emergency AC power system are
provided in the clarifying notes.

Unavailable hours are recorded for a component/train only when the
safety system is required to be available for service. Component/train
unavailable hours may be due to several causes that prevent a
component/train from performing its intended safety function when
required. These unavailable hours consist of the sum of the planned
and unplanned unavailable hours, and the fault exposure unavailable
hours, defined as follows:

\subparagraph{Planned Unavailable Hours}

Planned unavailable hours are hours that a component is not available
for service for an activity that is planned in advance. The beginning
and ending time of planned unavailable hours are known2.

Causes of planned unavailable hours include, but are not limited to,
the following:
\begin{itemize}
\item preventive maintenance or inspection requiring a component to be mechanically and/or electrically removed from service
\item planned support system unavailability causing a component or train of a monitored system to be unavailable (e.g. AC or direct current (DC) power, instrument air, service water, component cooling water, or room cooling)
\item surveillance testing, unless the testing configuration is automatically overridden by a valid starting signal or can be quickly overridden following such a signal, either by an operator in the control room or one stationed locally for that purpose.
\item any modification that requires a component to be mechanically
  and/or electrically removed from service
\end{itemize}

Station personnel sometimes question why planned unavailable hours are
included in the indicator definition. They point out that the
indicator is imposing a penalty for doing something that should be
positive in terms of results. The reason planned unavailable hours are
included, subject to additional provisions addressed in the Clarifying
Notes, is that the portions of system are unavailable during these
activities to perform their intended monitored safety function.

It is recognised that such planned activities can have a net beneficial effect in terms of reducing unplanned unavailability and fault exposure unavailable hours (as discussed further below). In fact, if planned activities are well managed and effective, fault exposure unavailable hours and unplanned unavailable hours are minimised.
Therefore, it is not necessarily desirable that planned unavailable
hours be avoided or minimised during periods when the system is
required to be available for service. Rather, the objective should be
to attain an overall indicator value that, while low, allows for
planned maintenance activities to help maintain system reliability and
availability consistent with safety analyses.

Stations with a high degree of redundancy in the design of the monitored systems may be able to avoid reporting planned unavailable hours under some circumstances, such as placing a spare or maintenance train out of service to perform planned maintenance.

\subparagraph{Unplanned Unavailable Hours}

Unplanned unavailable hours are the hours that a component is not
available for service for an activity that was not planned in
advance. The beginning and ending time of unplanned unavailable hours
are known3. Causes of unplanned unavailable hours include, but are not
limited to, the following:
\begin{itemize}
\item Corrective maintenance time following detection of a failed component (the time between failure and detection would be counted as fault exposure unavailable hours as discussed below).
\item Unplanned support system unavailability causing a component or train of a monitored system to be unavailable (e.g. AC or DC power, instrument air, service water, component cooling water,  or room cooling).
\item Human errors leading to component unavailability (e.g. valve or
  breaker mispositioning) – only the time to restore would be reported
  as unplanned unavailable hours.
\end{itemize}

\subparagraph{Fault Exposure Unavailable Hours}

The concept of fault exposure unavailable hours is extremely important
to the accuracy of the safety system performance indicator because it
reflects the amount of time that a component or train spends in an
undetected, failed condition. Three situations involving fault
exposure unavailable hours can occur.

In the first case, the failure’s time of occurrence and its time of
discovery are known. Examples of this type of failure include events
external to the equipment (e.g. a lightning strike, some
mispositionings by operators, or damage caused during test or
maintenance activities) that caused the component failure at a known
time. For these cases, the fault exposure unavailable hours are the
lapsed time between the occurrence of a failure and its time of
discovery.

In the second case, the failure is annunciated when it occurs. For
this case, there are no fault exposure unavailable hours because the
time of failure is the time of discovery. These failures include the
following:
\begin{itemize}
\item failure of a continuously operated component, such as the trip of an operating feedwater pump that is also used to fulfil a monitored system function, such as feedwater coolant injection in some BWRs
\item failure of a component while in standby that is annunciated in
  the control room, such as failure of control power circuitry for a
  monitored system
\end{itemize}

Failure of a standby component during a surveillance test is not
typically included in the second case, unless it is known with
certainty that the failure would not have occurred if the test had
been conducted at an earlier date. In other words, zero fault exposure
unavailable hours are only associated with standby component failures
that are caused by conditions unique to the particular test in which
the component fails.

For the third case, only the time of the failure’s discovery is known
with certainty. It is improper to assume that the failure occurred at
the time of discovery for these failures because the assumption
ignores what could be significant unavailable time prior to their
discovery. Fault exposure unavailable hours for this case must be
estimated. The basis for the estimation technique is rooted in
reliability theory. The theory states, for randomly occurring failures
in a group of components that are tested periodically, the average
time of occurrence will be one-half the time since the last successful
test. Thus, the value used to estimate the fault exposure unavailable
hours for this case is one-half the time that the component was
required to be operable since its last successful operation or test.

To improve the accuracy of reporting this important parameter,
utilities report the fault exposure unavailable hours for each
failure. The fault exposure unavailable hours are computed as half the
time the component was required to be available for service since the
last known test or operation unless specific justification is provided
that the failure is either of the first or second case listed (and
thus, the fault exposure unavailable hours are either known and
reported as such or else is not applicable). The data entry software
will provide prompts to help ensure this reporting is done as
accurately as possible.

\subparagraph{Hours system required}

The hours the system is required to be able to perform its safety
function. Unavailable hours are counted only for periods when a system
is required to be available for service.

The denominator of the unavailability calculation is the default value
for the hours the system was required to be able to perform its safety
function. The term ‘hours system required’ default value will be used
to calculate system unavailability and is as follows for each system:
\begin{itemize}
\item Emergency AC power system:  This value is estimated by the number of hours in the reporting period, because emergency generators are normally expected to be available for service during both unit/plant operation and shutdown.
\item BWR residual heat removal system:  This value is estimated by the number of hours in the reporting period, because the residual heat removal system is required to be available for decay heat removal at all times.
\item All other systems:  This default value is the number of critical
  hours during the reporting period, because these systems are usually
  required to be in service only while the reactor is critical, and
  for short periods during startup or shutdown. This data element is
  already provided as part of the station unplanned automatic scrams
  per 7,000 hours critical data.
\end{itemize}

\subparagraph{Number of required trains in the system}

The number of trains required to be in service is a fixed value and
does not change unless a design change is implemented that affects the
number and applicable technical specifications.  It is not the number
“installed”, but rather the number “required to be in service” during
normal full power operation.  (This is because additional spare
components or maintenance trains may be installed but are not required
to be in service during normal operation.

Spare components and maintenance trains are discussed elsewhere in the
chapter.  Refer to complete guidance for reporting unavailability and
fault exposure data).  The number of required trains is typically the
number required to be in service by applicable technical
specifications in order for the safety system to perform the system
functions monitored by the indicator while the unit is operating.

The number of required trains in the emergency AC power system is
determined as the number of emergency (diesel, gas turbine, or
hydro-electric) generators at the station that are installed to power
shutdown loads in the event of a loss of offsite power.

Data for the emergency AC power system is collected at the train
level. The emergency generator includes subsystems such as air start,
lube oil, fuel oil, cooling water, generator output breaker,
etc. However, for this safety system performance indicator,
unavailable hours are counted only when the emergency generator is
unavailable to start or load-run. For example, if a component fails in
one train of a redundant support system, the emergency generator is
still operable, and no unavailable hours are counted.

The number of required trains does not include the spare or
maintenance trains that may be installed for redundancy and easy
maintenance but are not required to be in service during normal plant
operations.

\subparagraph{Clarifying Notes}

Some components in a system may be common to more than one train, in
which case the effect of the performance (unavailable hours) of a
common component is included in all affected trains.

Except as noted, the following clarifying notes are applicable to all
the safety systems monitored for this indicator.

\subparagraph{Data Sources}

Sources for identifying component unavailable hours can be obtained
from component failure records, control room logs, event reports,
maintenance work orders, etc. Preventive maintenance and surveillance
test procedures may be helpful in determining if activities performed
using these procedures cause components to be made unavailable and in
identifying the frequency of these maintenance and test activities.

\subparagraph{Component Scope Exceptions}

Components (e.g. annunciators, transmitters, etc.) that do not affect
the monitored function of a system's principal components should not
be included in the scope of components monitored for this
indicator. For example, a pressure transmitter that provides only an
indication of pressure would not be included. However, a pressure
transmitter that could prevent a pump from starting would be
monitored.

\subparagraph{Component Failure Exceptions}

Component malfunctions or operating errors that did not prevent the
component from being restored to normal operation within a few minutes
from the control room, and did not require corrective maintenance or
significant problem diagnosis, are not counted as failures.

A small oil, water and steam leak that would not preclude safe
operation of the component during an operational demand of the safety
system is not counted as a component failure.

Unavailable hours (planned, unplanned and fault exposure) are not
reported for the failure of certain ancillary components unless the
safety function of a principal component (e.g. pump, valve, emergency
generator) is affected. Such ancillary components include equipment
associated with control, protection, and actuation functions; power
supplies; lubricating subsystems; etc. For example, if three pressure
switches arranged in a two-out-of-three logic provide low suction
pressure protection for a BWR residual heat removal pump, unavailable
hours would not be counted for a failure of only one of these pressure
switches because the single failure would not affect operability of
the pump.

\subparagraph{Fault Exposure Unavailable Hours}

To provide sufficient information to properly assess fault exposure
unavailable hours, at least two tests or operations of system
components must be performed between overhauls or major maintenance of
the components. The first test may be associated with the
post-maintenance demonstration that the system is available for
operation at the beginning of the time period during which the system
is required to be available. At least one more test is required near
the end of this period, prior to the next overhaul or major
maintenance, to determine whether the components in the system have
remained available (i.e. unfailed) throughout the period.

If a second test is not conducted to verify the ‘as-found’ operability
of the system components prior to maintenance, fault exposure
unavailable hours should be estimated and annotated.

When a failed or mispositioned component is discovered during an
inspection or by incidental observation (without being tested), fault
exposure unavailable hours should still be reported.

A component is not unavailable if it is capable of performing its
safety function. For example, if a normally open valve is found failed
in the open position and this is the position required for the system
to perform its function, fault exposure unavailable hours would not be
counted for the time the valve was in a failed state. However,
unplanned unavailable hours would be counted for the repair of the
valve, if the repair required the valve to be closed or the line
containing the valve to be isolated, and this degraded the full
capacity or redundancy of the system.

Fault exposure unavailable hours are not counted for a failure to meet
design or technical specifications, if engineering analysis determines
the component was capable of performing its safety function during an
operational event. For example, if an emergency generator fails to
reach rated speed and voltage in the precise time required by
technical specifications, the generator is not considered unavailable
if the test demonstrated that it would start, load, and run as
required in an emergency.

When a component failure is detected, the time since the last
successful test or operational demand may include some time when the
system was not required for service. In this case, the fault exposure
unavailable hours are estimated as one-half the time the system was
required to be available since the last successful test or
operation. For example, if a PWR high pressure injection pump is
discovered to be failed 24 days after the last successful pump test,
and a 10-day outage (when operability of the pump is not required)
occurred between the last successful test and discovery of the
failure, fault exposure unavailable hours for the pump are computed as
follows:  1/2 x (24 - 10) days x (24 hours/day) = 168 hours.

When both the time of component failure and the time of failure discovery are known, fault exposure unavailable hours are calculated as the time the component was required to be available for service during the period between the time of failure and the time of discovery.
The fault exposure unavailable hours associated with a component
failure may include unavailable hours covering several reporting
periods (e.g. several quarters). In this case, the fault exposure
unavailable hours should be assigned to the appropriate reporting
periods. For example, if a failure is discovered on the 10th day of a
quarter and the estimated number of unavailable hours is 300 hours,
then 240 hours should be counted for the current quarter and 60
unavailable hours should be counted for the previous quarter. Note:
This will require an update of the previous quarter's data.)

\subparagraph{Multiple Components Unavailable}

When two (or more) components in a system are unavailable concurrently
due to failures, the unavailable hours (unplanned and fault exposure
unavailable hours) for each component are counted separately when
determining the sum of the component unavailable hours for the
system. However, if following a component failure, one or more
additional components in the system are electively removed from
service for preventive maintenance, the unavailable hours on these
additional components are not counted except as follows:
\begin{itemize}
\item If the equipment outage time has to be extended to complete the elective maintenance work, then the additional hours should be counted as planned unavailable hours.
\item If a failure is found on a component electively removed from
  service, the fault exposure unavailable hours and any additional
  unplanned unavailable hours should be counted for that component in
  addition to the hours attributed to the original component failure.
\end{itemize}

When one component in a safety system is unavailable due to planned
activities (e.g. preventive maintenance), other components in the
system may be removed from service for planned activities without
counting additional unavailable hours if removal of these components
from service does not further degrade the full capacity or redundancy
of the system. For this case, the planned unavailable hours are
counted from the time the first component is removed from service
until the last component is restored to service.

Unavailable hours are not counted for components serving as isolation
devices for other components that are removed from service. For
example, if several valves are closed to isolate a pump for preventive
maintenance, then unavailable hours for the isolation valves would not
be counted. (Unavailable hours for the pump would be counted.)

\subparagraph{Support Systems}

If the unavailability of a support system causes a component that is
monitored in one of the safety systems to be unavailable, then the
hours the support system was unavailable are counted against the
monitored component. The support system unavailability reported should
be the actual unavailability of the support system function to the
monitored component.

Fault exposure unavailable hours due to support system unavailability
should not be reported, assuming the support system is continuously
running. If the support system is dedicated to the monitored system
and is normally in standby, it should be included as part of the
monitored system scope. In this case, fault exposure unavailable hours
caused by a failure in the support system should be reported, since it
involves a monitored component. Failures of continuously-operated
support systems do not contribute to fault exposure unavailable hours
in monitored systems.

If a continuously operating support system failure causes multiple
components to become unavailable, the unplanned unavailable hours
should be reported as indicated in the previous section.  Fault
exposure unavailable hours would not be reported in this case.

Emergency AC power is not considered to be a support system because this system is monitored separately for this indicator. Additionally, the monitored systems are considered to be available as long as a normal power source or an emergency power source for the systems is available.
Unavailable hours are also counted for the unavailability of support
systems that maintain required environmental conditions in rooms in
which safety system components are located if the absence of those
conditions is determined to have rendered monitored components
unavailable.

In some instances, unavailability in a monitored system that is caused
by unavailability of a support system used for cooling need not be
reported if cooling water from another source can be
substituted. Limitations on the source of the substituted cooling
water are as follows:
\begin{itemize}
\item For monitored fluid systems with components cooled by a support system, where both the monitored and the support system pumps are powered by a class 1E (i.e. safety grade or an equivalent)electric power source, cooling water supplied by a pump powered by a normal (non-class 1E – i.e. non-safety grade) electric power source may be substituted for cooling water supplied by a class 1E electric power source, provided that redundancy requirements to accommodate single failure criteria for electric power and cooling water are met. Specifically, unavailable hours must be reported when both trains of a monitored system are being cooled by water provided by a single cooling water pump or by cooling water pumps powered by a single class 1E power (safety grade) source.
\item For emergency generators, cooling water provided by a pump
  powered by another class 1E (safety grade) power source can be
  substituted, provided a pump is available that will maintain
  electrical redundancy requirements such that a single failure cannot
  cause a loss of both emergency generators.
\end{itemize}

\subparagraph{Installed Spares}

For systems that have installed spare components, unavailable hours
are not counted in certain situations. An ‘installed spare’ is a
component (or train of components) that is used as a replacement for
other equipment to allow for the removal of equipment from service for
preventive or corrective maintenance without incurring a limiting
condition for operation (where applicable) or violating the single
failure criterion. To be an ‘installed spare,’ a component must not be
required in the design basis safety analysis for the system to perform
its safety function.

Components that are required as backup in case of equipment failure to
allow the system to meet redundancy requirements or the single failure
criterion (e.g. swing components that automatically align to different
trains or units) are not installed spares.

Unavailable hours for an installed spare are counted only if the
installed spare becomes unavailable while serving as replacement for
another component. This includes planned and unplanned unavailable
hours and fault exposure unavailable hours.

Planned unavailable hours (e.g. preventive maintenance) and unplanned
unavailable hours (e.g. corrective maintenance) are not counted for a
component when an installed spare has been placed in service to
perform the component's safety function.

Fault exposure unavailable hours associated with component failures or
human errors are always counted, even if the failed component is
replaced by an installed spare while it is being repaired. For
instance, a pump in a high pressure safety injection system (that has
an installed spare pump) fails its quarterly surveillance
test. Unavailable hours reported for this failure would include the
time needed to substitute the installed spare pump for the failed pump
(unplanned unavailable hours) and one-half the time since the last
successful operation of the failed pump (fault exposure unavailable
hours).

Installed spares are not counted as principal pumps or emergency
generators for the purpose of determining the number of trains.

\subparagraph{Safety Systems with Extra (Redundant) Trains}

Some power plants have safety systems with extra trains of components
to allow preventive maintenance to be carried out with the unit at
power without violating the single failure criterion (when applied to
the remaining trains). That is, one of the remaining trains may fail,
but the system can still achieve its safety function as required by
the design basis safety analysis. Such systems are characterised by a
large number of trains (usually a minimum of four, but often more).

Fluid systems that have such extra trains generally must meet design
bases requirements with one train in maintenance and a single failure
of another train.

For purposes of the safety system performance indicator, plants that
have systems with an extra train (maintenance train) as defined above,
that may be removed from service for an unlimited time, may avoid
reporting both planned and unplanned unavailability for one such train
at a time. Fault exposure unavailable hours associated with failures
involving these trains are always counted and reported, irrespective
of whether the failure occurred while the train was in standby service
or in maintenance.

The following examples will help illustrate the fluid system
requirements in order to benefit from this provision:
\begin{itemize}
\item Any system containing three 50\% (flow rate and/or cooling capacity) trains would not meet the requirements since full design flow rate would not be available with one train in maintenance and one train failed (single failure criterion).
\item Similarly, a system with four 50\% trains or three 100\% trains
  may meet the criterion, assuming the system design flow rate and
  cooling requirements can be met during a design basis accident
  anywhere within the reactor coolant or secondary system boundaries,
  including unfavourable locations of LOCAs and feedwater line
  breaks. This statement is not intended to set new design criteria,
  but rather, to define the level of system redundancy required if
  reporting of unavailable hours on a redundant train is to be
  avoided.
\end{itemize}

\subparagraph{Systems Required to be in Service at All Times}

The emergency AC power system and the BWR residual heat removal (RHR)
system are normally required to be in service at all times. However,
planned and unplanned component unavailable hours are not reported
when certain components (for example, emergency generator, RHR pump)
are removed from service (e.g. for preventive maintenance, corrective
maintenance, or overhauls).

Emergency AC Power System - when a unit (or units) is/are shut down,
one emergency AC power train at a time may be removed from service
without incurring planned or unplanned unavailable hours under the
following conditions:
\begin{itemize}
\item For a single or multi-unit station with all units shut down, one emergency generator (EG) at a time may be electively removed from service without reporting planned and unplanned unavailable hours providing that at least one operable EG is available to supply emergency loads.
\item For a multi-unit station with one unit shut down and all other
  units operating, one EG at a time may be electively removed from
  service without reporting planned and unplanned unavailable hours
  providing that both of the following criteria are satisfied:
  \begin{itemize}

  \item The EG removed from service is associated primarily with a unit that is shut down.

  \item Removal of the EG from service has little effect on the safety of the operating units (i.e. required emergency loads for each operating unit can be met, even when accounting for the single failure of an operable EG), and there is still an operable emergency generator available to the shut down unit.
EGs may be tested without reporting planned and unplanned unavailable
hours provided that the EGs remain available for use during the test
(i.e. the testing configuration is overridden by a valid automatic
start signal or can be quickly overridden by control room operators or
operators stationed locally for that specific purpose).
\end{itemize}

\item When the reactor is in shutdown, those systems or portions of systems that provide shutdown cooling can be removed from service without incurring planned or unplanned unavailable hours under the following conditions:
\item Those portions of the shutdown cooling system associated with one heat exchanger flow path can be taken out of service without incurring planned or unplanned unavailable hours provided the other heat exchanger flow path is available (including at least one pump) and an alternate, closed-cycle, forced means of removing core decay heat is available. The alternate means of decay heat removal need not be safety-related, but must have been determined to be capable of handling the decay heat load.
\item With fuel still in the reactor vessel, when the decay heat load
  is so low that forced recirculation for cooling purposes, even on an
  intermittent basis, is no longer required (ambient losses are enough
  to offset the decay heat load), component planned or unplanned
  unavailable hours are not reportable.
  \begin{itemize}
  \item When the reactor is defuelled, component planned or unplanned unavailable hours are not reportable.
  \item When the bulk reactor coolant temperature is less than 200W F,
    those systems or portions of systems whose sole function is to
    provide suppression pool cooling may be removed from service
    without incurring planned or unplanned unavailable hours.
  \end{itemize}

\item When portions of a system provide both the shutdown cooling and the suppression pool cooling function, the most limiting set of reportability requirements should be used (i.e. unavailable hours and required hours are reported whenever at least one function is required.)
\item For these systems, component fault exposure unavailable hours are always counted, even when portions of the system are removed from service as described above.
\item When the plant is operating, selected components that help
  provide the shutdown cooling function of the RHR system are normally
  deenergised or racked out. This does not constitute an unavailable
  condition for the trains that provide shutdown cooling, unless the
  deenergised components cannot be placed back into service before the
  minimum time that the shutdown cooling function would be needed
  (typically the time required for a plant to complete a rapid cool
  down, within maximum established plant cool down limits, from normal
  operating conditions).
\end{itemize}

\subparagraph{Additional Notes for Emergency Generators (EG)}

Fault exposure unavailable hours should not be counted for failures of
an EG to start or load-run if the failure can be definitely attributed
to reasons listed under Component Failure Exceptions above, or to any
of the following:
\begin{itemize}
\item Spurious operation of a trip that would be bypassed in the loss of offsite power emergency operating mode (e.g. high cooling water temperature trip that erroneously tripped an EG although cooling water temperature was normal).
\item Malfunction of equipment that is not required to operate during the loss of offsite power emergency operating mode (e.g. circuitry used to synchronise the EG with offsite power sources, but not required when offsite power is lost).
\item A failure to start because a redundant portion of the starting
  system was intentionally disabled for test purposes, if followed by
  a successful start with the starting system in its normal alignment.
\end{itemize}

When determining fault exposure unavailable hours for a failure of an
EG to load-run following a successful start, the last successful
operation or test is the previous successful load-run (not just a
successful start). To be considered a successful load-run operation or
test, an EG load-run attempt must have followed a successful start and
satisfied one of the following criteria:
\begin{itemize}
\item a load-run of any duration that resulted from a real (e.g. not a test) manual or automatic start signal
\item a load-run test that successfully satisfied the plant's load and duration test specifications
\item other operation (e.g. special tests) in which the emergency
  generator was run for at least one hour with at least 50\% of design
  load
\end{itemize}

\subparagraph{Unit-based AC emergency power systems approach}4
A multi-unit site reports the unavailable hours of all AC emergency
power systems for each individual unit.

If, according to design5 information, plant has unit-based and
station-based AC emergency power systems, DES allows insertion of this
information separately.

All information related to “Emergency AC Power” in DES will be
collected unit based in a similar way as data elements for SP1 and
SP2.

\subsubsection{Fuel Reliability}

\paragraph{Data Elements}

The following data is required for this data category:

\subparagraph{AGR (Advanced Gas Cooled Reactor)}

The monthly average failed fuel fraction as determined from gamma
spectrometer measurements of the amount of Xe-133, Xe-135, Xe-138,
Kr-88, Kr-87 and Kr-89 (or as many of these as possible) in the
primary coolant.

All AGR's have a gaseous activity monitor. This consists of a gamma
spectrometer, a computerised data logger, and an analysis programme,
‘AYPLOT’. The equivalent uranium contamination level and the failed
fuel fraction are derived for each 24 hour period.

The data to report is the median of the unit values for that member.

The calculation assumes the coolant specific activity is in
equilibrium with the activity release rate; more than ten days after a
major power change the coolant activity should normally be within 25\%
of equilibrium. With no failed fuel present in the reactor, the failed
fuel fraction should be calculated to be zero. However, statistical
variations in the data could give both positive and negative small
values for the failed fuel fraction.

As a negative failed fuel fraction has no meaning, negative values for the failed fuel fraction should be recorded as zero. Thus, a mean value of the failed fuel fraction averaged over any period will be biased to a positive value; this bias will correspond to about half the true standard deviation.
Thus, a measured value of failed fuel fraction of less than 2-4 x 10-4
has no real significance at the 5\% confidence level.

The equilibrium coolant specific activity is modified by the coolant
leakage rate; this is typically less than one tonne of CO2 per day,
corresponding to a loss rate of less than 2 x 10-7s-1. This loss rate
will not significantly alter the equilibrium coolant activity of
Xe-133 and thus will not significantly alter the calculated value of
failed fuel fraction. Higher loss rates due to reactor purging (to
reduce the Ar-41 activity for example) may affect the equilibrium
coolant activity of Xe-133 and thus the calculated value of the failed
fuel fraction. The calculation does not account for coolant loss rate.

\subparagraph{BWRs (Boiling Water Reactors) and HWLWRs (Heavy Water Moderated Light
Water Cooled Reactors)}

The monthly average reactor coolant activity rates (Becquerels/second or microcuries/second) of the krypton-85m, krypton-87, krypton-88, xenon-133, xenon-135, and xenon-138 isotopes; the average reactor power (\%) at the time activities are measured; and linear heat generation rate (kilowatts per meter or foot).
The linear heat generation rate used as the basis for normalisation
(16.5 kilowatts per meter or 5.0 kilowatts per foot).

An example of the data collected for BWRs/HWLWRs is provided in the Attachment.

\subparagraph{MAGNOX}

For each month, the sum of the times (hours) that each channel in the
reactor contained failed fuel during the time that the reactor was
operating, as indicated by the appropriate system.

The conversion from Becquerels (Bq) to microcuries ($\mu Ci$) is 1
microcurie = 37,000 Becquerels.

\subparagraph{PWRs (Pressurised Water Reactor), VVERs
  (Russian-designed PWR) and PHWRs (Pressurised Heavy Water Reactor)}

The monthly average reactor coolant activity (Becquerels/gram or
microcuries/gram) for iodine-131 and iodine-134; the actual
purification rate constant (seconds -1); the average reactor power
(\%) at the time activities are measured; and the linear heat
generation rate (kilowatts per meter or foot).

For PHWRs, refuelling a channel containing defective fuel causes a
rapid increase in iodine activity due to power changes in the
channel. Activities measured under these transient conditions should
not be included when determining monthly averages.

For PWRs, VVERs and PHWRs, iodine-134 activities may not be available at some units. In this case, a value of zero (0.0) should be used for iodine-134 activity. The resulting indicator value will be conservative in that it will overestimate the coolant iodine-131 activity due to defective fuel.
An example of the data collected for PWRs and PHWRs is in the
Attachment \ref{VVER_CPI}.

\subparagraph{RBMKS (Russian-designed Light Water Cooled Graphite
  Reactors)}

The number of fuel assemblies with leaking fuel elements offloaded
from the reactor each month and the number of effective full power
days of reactor operation during each month.

\subparagraph{FBR (Fast Breeder Reactor)}

The monthly average data cover gas activity rates (Becquerel/cubic
centimetre (cc) or micro curies/cc) of the Xenon-133, Xenon-138;
reactor power (percent); and linear heat generation rate (kilowatts
per meter or foot)

\subparagraph{Clarifying Notes}

Values reported for measure activities should never be a value lower
than the ‘minimum detectable activity’ (MDA) value capability of the
measuring instrumentation. If the value is below MDA the MDA value
should be input.

Steady state is defined as continuous operation for at least three
days prior to data collection for BWRs/HWLWRs and PWRs, two days for
PHWRs, and ten days for AGR and Magnox's at a power level that does
not vary more than + 5\%.   In order to obtain an indicator value for
a month, the steady state power at which data is collected must be
85\% or greater for BWR and PWR designs. This ensures appropriate
indicator accuracy. For months where no period of steady state power
was 85\% or greater, the highest steady state power achieved should be
reported. In this case, reporting of additional fuel reliability
indicator data for that month can be omitted since it will not be used
in calculating an FRI value.

To calculate FRI the recent available data is used. If a valid result
for a BWR/PWR/PHWR cannot be calculated because the reported power
level is below the required value, the result in the PI REPORTS will
be coded “P” on applicable reports or spreadsheets. Contact your
regional center for further information. If the power requirements of
this indicator cannot be met then “DN” (DOWN) into DES values field
has to be entered. See online help on the PI webpage.

\subsubsection{Chemistry}

\paragraph{Data Elements}

The following data is required:

\subparagraph{BWRs, and heavy water moderated light water cooled reactors (HWLWRs)}
\begin{itemize}
\item reactor water chloride
\item reactor water sulphate
\item final feedwater iron
\end{itemize}

\subparagraph{LWCGRs (RBMKs)}
\begin{itemize}
\item reactor water total conductivity
\item reactor water chloride
\item final feedwater iron
\end{itemize}

\subparagraph{PWRs with recirculating steam generators and VVERs}\footnote{ Actual parameters used vary with plant design and chemistry regime. See Calculations section.}
\begin{itemize}
\item steam generator blowdown chloride
\item steam generator blowdown cation conductivity
\item steam generator blowdown sulphate
\item steam generator blowdown sodium
\item final feedwater iron
\item final feedwater
\item condensate dissolved oxygen
\item steam generator molar ratio target range (by reporting the upper and lower range limits (as ‘from’ and ‘to’ values when using molar ratio control)
\item steam generator actual molar ratio (if reporting molar ratio control data)
\item feedwater oxygen
\item feedwater pH value at 270ºC
\end{itemize}

\subparagraph{PWRs with once through steam generators}
\begin{itemize}
\item final feedwater chloride
\item final feedwater sulphate
\item final feedwater sodium
\item final feedwater iron
\item final feedwater copper
\end{itemize}

\subparagraph{AGRs}
\begin{itemize}
\item final feedwater chloride
\item final feedwater sulphate
\item final feedwater sodium
\item final feedwater iron
\item final feedwater acid conductivity
\end{itemize}

\subparagraph{Gas cooled reactors (except those Magnox units with drum
  (recirculating)  boilers)}
\begin{itemize}
\item final feedwater chloride
\item final feedwater sulphate
\item final feedwater sodium
\item final feedwater iron
\item final feedwater dissolved oxygen\footnote{Oxygen target on
    reactors using feedwater oxygen dosing, where: the oxygen target =
    O2 Target = [| T - O2 |]/ T, where: O2 = the measured dissolved
    oxygen concentration in the final feedwater and, T = The target final feedwater O2 concentration, and | T - O2 | = the absolute value of the T - O2}
\end{itemize}

\subparagraph{Gas cooled Magnox reactors with drum (recirculating)
  boilers}
\begin{itemize}
\item final feedwater cation conductivity
\item final feedwater dissolved oxygen
\item an alkali deficiency term based on the difference between the
\item alkalinity of the boiler water as measured (either equivalent
  sodium hydroxide or phosphate density) and the ‘Target’
  alkalinity. Thus, Alkali deficiency = T - Alkalinity/ T, where: T =
  the boiler water alkalinity corresponding to the minimum
  concentration of non-volatile alkali needed to achieve the required
  level of protection against corrosion. This term is set to zero if
  alkalinity is > T, reflecting the fact that alkalinity is more than
  adequate.
\end{itemize}

\subparagraph{Pressurised heavy water reactors (PHWRs)}
\begin{itemize}
\item Inconel-600 or Monel tubes
\begin{itemize}
\item steam generator blowdown chloride
\item steam generator blowdown sulphate
\item steam generator blowdown sodium
\item final feedwater iron
\item final feedwater copper
\item final feedwater dissolved oxygen
\end{itemize}
\item Incoloy-800 tubes
\begin{itemize}
\item steam generator blowdown chloride
\item steam generator blowdown sulphate
\item steam generator blowdown sodium
\item final feedwater iron
\item final feedwater dissolved oxygen
\end{itemize}
\item PHWRs on molar ratio control
\begin{itemize}
\item steam generator blowdown chloride
\item steam generator blowdown sulphate
\item final feedwater iron
\item final feedwater copper
\item feedwater dissolved oxygen
\item steam generator molar ratio target range (by reporting the upper and lower range limits (as ‘from’ and ‘to’ values)
\item steam generator actual molar ratio
\end{itemize}
\end{itemize}

\subparagraph{FBRs with evaporators and super heaters in steam
  generators}
\begin{itemize}
\item steam generator blow down chrolide
\item steam generator blow dawn sodium
\item final feed water copper
\item final feedwater iron
\end{itemize}

\paragraph{Data collection}

For each chemistry parameter with the exception of final feedwater
iron and copper, the quarterly value is the average (mean) of the more
frequent (daily or weekly) measurements for that parameter during the
quarter, subject to the following conditions:
\begin{itemize}
\item The daily measured value for a parameter is based on the time-weighted
average value for the parameter. The daily time-weighted average is
defined as follows:

$$ \text{Time-weighted average} = \frac{\sum{(V_i \cdot T_i)}}{\sum{T_i}} $$ for the day

Where:	Vi =	the recorded value of a parameter

Ti  = 	the time from the measurement of Vi until the next recorded
measurement of the parameter

A day is a 24-hour period extending from midnight to midnight, and the
value of a parameter at the beginning of a day is considered to be the
same as the last measured value from the preceding day. For
consistency of comparison and accuracy of data, it is recommended that
on-line monitor data be used where available. For BWRs, the daily
values are the time-weighted average values for each parameter. For
PWRs, the daily values are the averages of the time-weighted average
values for each measurement location (steam generator blowdown or
feedwater train). For example, the value for the day for blowdown
sodium is determined by adding the time-weighted average values of the
parameter for each steam generator, and dividing by the number of
steam generators. A similar time-weighted value can be calculated for
parameters measured at other frequencies (e.g. weekly).

\item Final feedwater iron and copper concentrations will be obtained
  based on a seven-day integrated value. For plants whose feedwater
  has a high concentration of iron or copper, this may not be feasible
  and integrated values obtained on a frequency of less than seven
  days may be used for reporting the individual components of the
  chemistry indicator. The quarterly value is the average (mean) of
  the weekly (or other frequency as determined by actual water
  chemistry) measurements. The parameter value is based on the
  time-weighted average value as defined by the following:

$$ \text{Time-weighted average} = \frac{\sum{(V_i \cdot T_i)}}{\sum{T_i}} $$
for the period.

Use whole days in the calculation.

\item The unit values for the indicator are determined using the
  following parameters:

Please pay your attention that all limits are described in the CPI
section.
\subparagraph{BWRs and HWLWRs:}
\begin{itemize}
\item RW Cl		=	Average reactor water chloride (ppb)
\item RW SO4	=	Average reactor water sulphate (ppb)
\item FW Fe		=	Average feedwater iron (ppb)
\item LVx		=	Limiting value for that parameter
\end{itemize}

Unit values for periods of greater than one quarter are the weighted
average of quarterly values in the period. The quarterly values are
weighted using the number of days above 10\% power in each quarter.

\subparagraph{LWCGRs (RBMKs)}
\begin{itemize}
\item RW K		=	Average reactor water total conductivity (us/cm)
\item RW Cl		=	Average reactor water chloride (ppb)
\item FW Fe		=	Average feedwater iron (ppb)
\item LVx		=	Limiting value for that parameter
\end{itemize}

Unit values for periods of greater than one quarter are the weighted
average of quarterly values in the period. The quarterly values are
weighted using the number of days above 30\% power in each quarter.

\subparagraph{PWR RECIRCULATING STEAM GENERATORS, VVERs AND PHWR
  PLANTS WITH INC0NEL-600 OR MONEL TUBES:  (PWRs with I-800 steam
  generator tubes should refer to separate formula provided further
  below)}
\begin{itemize}
\item SG Cl		=	Steam generator blowdown chloride
\item SG SO4		=	Steam generator blowdown sulphate
\item SG Na		=	Steam generator blowdown sodium
\item FW Fe		=	Final feedwater iron
\item FW Cu		=	Final feedwater copper
\item *O2		=	Final feedwater dissolved oxygen (steam generator blowdown cation; only for PHWRs)
\item LVx		=	Limiting value for that parameter
\item SG Conductivity = 	Steam generator Conductivity (only for
  VVER).
\end{itemize}

Unit values for periods of greater than one quarter are the weighted
average of quarterly values in the period. The quarterly values are
weighted using the number of days above 30\% power in each quarter.

\subparagraph{PWR RECIRCULATING STEAM GENERATORS OR PHWR REACTORS ON
  MOLAR RATIO CONTROL:  (PWRs with I-800 steam generator tubes should
  refer to separate formula provided further below)}
\begin{itemize}
\item SG SO4		=	Steam generator blowdown sulphate
\item SG Na		=	Steam generator blowdown sodium\footnote{steam generator blowdown chloride for PHWR design}
\item FW Fe		=	Final feedwater iron
\item FW Cu		=	Final feedwater copper
\item TMR deviation	=	Absolute value of target molar ratio range (upper or lower end [the ‘from’ or
        ‘to’ value - whichever is closest] minus the actual molar ratio)
      \item O2		=	Final feedwater dissolved oxygen (only for PHWR)
      \item LVx		=	Limiting value for that parameter
      \end{itemize}

      Unit values for periods of greater than one quarter are the
      weighted average of quarterly values in the period. The
      quarterly values are weighted using the number of days above
      30\% power in each quarter.

\subparagraph{PWR RECIRCULATING STEAM GENERATORS AND PHWR WITH I-800
  TUBES}
\begin{itemize}
\item SG Ka		=	Steam generator blowdown cation conductivity\footnote{only applicable for PWR with I-800 steam generator tubes}
\item SG Cl		=	Steam generator blowdown chloride
\item SG SO4		=	Steam generator blowdown sulphate
\item SG Na		=	Steam generator blowdown sodium
\item FW Fe		=	Final feedwater iron
\item O2		=	Condensate dissolved oxygen\footnote{final feedwater dissolved oxygen for PHWR plants}
\item LVx		=	Limiting value for that parameter
\end{itemize}
Unit values for periods of greater than one quarter are the weighted
average of quarterly values in the period. The quarterly values are
weighted using the number of days above 30\% power in each quarter.

\subparagraph{PWR RECIRCULATING STEAM GENERATORS ALTERNATE GROUP}
\begin{itemize}
\item SG Ka		=	Steam generator cation conductivity
\item SG Na		=	Steam generator sodium
\item FWO2		=	Feedwater oxygen
\item FW pH270 	=	Feedwater pH at 270ºC*
\item LVx		=	Limiting value for that parameter
units of this group not monitoring pH use the iron parameter term FW Fe/LVx (in place of FWpH270 term)
\item FW Fe		=	Feedwater iron*
\end{itemize}

Unit values for periods of greater than one quarter are the weighted
average of quarterly values in the period. The quarterly values are
weighted using the number of days above 30\% power in each quarter.

\subparagraph{ONCE-THROUGH STEAM GENERATORS}
\begin{itemize}
\item FW Cl		=	Final feedwater chloride
\item FW SO4		=	Final feedwater sulphate
\item FW Na		=	Final feedwater sodium
\item FW Fe		=	Final feedwater iron
\item FW Cu		=	Final feedwater copper
\item LVx		=	Limiting value for that parameter
\end{itemize}

Unit values for periods of greater than one quarter are the weighted
average of quarterly values in the period. The quarterly values are
weighted using the number of days above 30\% power in each quarter.
\end{itemize}

\subparagraph{Clarifying Notes}
Include only samples taken above 10\% reactor power for BWR plants,
30\% reactor power for PWR/VVER plants and 40\% reactor power for FBR
and other plant designs.

If a sample is missed for a day, the data values should be carried
over from the previous day, unless power is at or below 10\% reactor
power for BWR plants, or 30\% reactor power for PWR plants, or 40\%
reactor power for FBR plants and other plant designs.

Values that are less than the lowest measurable level should be
reported as equal to the limit of quantification. The limit of
quantification is defined as the lowest concentration of a parameter
that must be present before the measurement of that concentration is
considered a quantitative result. In the other words, the
concentration above which one can state with a known degree of
confidence that a pollutant (or other material) is present at a
specific concentration in the sample tested. It is determined by
performing greater than 20 replicates of a blank analysis for a
specific parameter. For example, 20 or more analyses of a blank for a
chloride test.

Chemical parameters should be expressed in units of parts per billion
(ppb = 1E-9).

For those PWRs adding chloride to control molar ratio, a slightly
different indicator equation is used that removes the chloride term,
and replaces it with a performance measure of the molar ratio
control. This term is site specific and uses the absolute value of the
target molar ratio deviation as an indicator parameter. In the
electronic data reporting software, this range is indicated by
reporting the lower and upper range values. The Target Molar Ratio
Deviation term used in the calculation is the absolute value of the
difference between the upper or lower limit of that range (whichever
is closer) and the actual molar ratio. The chloride term is removed at
these stations as it would no longer be consistent with the molar
ratio operating philosophy selected by the station. In recognition of
the increased importance of the molar ratio in operational chemistry
performance for those stations selecting this control regimen, a
weighting factor of 5.0 has been applied to the target molar ratio
term in the indicator equation. Refer to the Calculations section for
further explanation.

For BWR plants, iron has been assigned a weighting factor of 1/2 as
compared to the other parameters of the indicator to reflect its
relative lower level of importance as compared to chloride and
sulphate. Iron has not been directly related to failures of vessel
internal components from corrosion, and has not played a significant
role in recent fuel failures. However, iron transported to the reactor
has been shown to have significant effect on station dose rates, and
impacts chemistry controls of stations that inject zinc. Also,
feedwater iron is an indicator of possible corrosion problems in the
steam cycle systems.

For FBR plants, controlled chemical elements and the target molar
ratios are specified as indicators of possible corrosion problems
including an occurrence of stress corrosion cracking (SCC) in steam
generators with chrome molybdenum steel tube and austenitic stainless
steel tube.

Parameter data should not be omitted for any reason not specifically
addressed in this guidance. For examples, data should not be omitted
from a reporting period just because there were a low number of days
meeting power requirements in that reporting period; also, data should
not be omitted because the unit failed to reach stable conditions.

Different parameters are used by different pressurised water reactors
due to different chemistry monitoring and control practices.

To calculate CPI at least 50\% of available accurate data is used. If there is not enough accurate available data during reporting period the ‘P’ indicator status should be reported. P status means value was not qualified because reactor power was below required level.

\subsection{Radiation Protection}
\subsubsection{Data Elements}

These data elements are obtained quarterly6. If TLD or film badge
information is not available for the current quarter, the data may
include direct reading dosimeter data, such as pocket dosimeters. If
TLD or film badge data becomes available later, a data revision should
be provided. The following data is required to determine each unit's
value for this indicator:
\begin{itemize}
\item External whole body exposure: total whole man-Sieverts (or man-rem) exposure from external exposure for the current quarter
\item calculated internal whole body exposure from internally
  deposited radioactive material for any nuclide resulting from plant
  activities for the current quarter
\end{itemize}

\subparagraph{Clarifying Notes}

Data elements (external plus internal) should be tracked on a per-unit
basis. Specific unit-related dose, including outages-related dose,
shall be reported on a per unit basis for the applicable unit. For
multi-unit stations that do not track data elements separately for
each unit, unit dose values during normal operations may be estimated
by dividing the station data by the number of operating units at the
station. Dose received at a particular unit shall not be divided
equally among the units at the multi-unit station.

All personnel exposures should be reported as accurately as possible,
preferably within an accuracy of 0.00001 Sieverts (one millirem). The conversion from Sieverts (Sv) to rem is 1 Sievert = 100 rem.

\subsection{Personnel Safety}
\subsubsection{Plant Personnel}
\paragraph{Data Elements}
The following data is required:

Restricted Work

\begin{itemize}
\item  Accidents: the number of restricted-time accidents involving days of restricted work for utility personnel assigned to the station
\item Lost-time accidents: the number of lost-time accidents involving days away from work for utility personnel assigned to the station
\item Work-related Fatalities: the number of work-related fatalities for utility personnel assigned to the station
\item Total hours worked by station personnel: the number of man-hours
  worked by utility personnel assigned to the station
\end{itemize}

An example of the data collected for this performance indicator is
provided in the Attachment \ref{ISA}.

\subparagraph{Clarifying Notes}

To be counted, an accident must result in one or more days away from
work, one or more days of restricted work, or both (excluding the day
of the accident), or result in a fatality.

\subparagraph{Restricted work activity includes one of the following:}
\begin{itemize}
\item assignment to another job on a temporary basis
\item working at a permanent job less than full time
\item working at a permanently assigned job but not able to perform
  all duties normally connected with the job because of the injury
\end{itemize}

A single accident should only be reported in the most serious of the
three categories, with fatalities being the most serious, lost-time
accidents being next most serious, and restricted work accidents being
least serious. Thus, an accident resulting in both lost time and
restricted work would be reported during the quarter it occurred as a
lost-time accident.
\begin{itemize}
\item Any accident that results in a work-related fatality shall be included as part of the indicator. Although not considered lost-time accidents, these incidents represent the most serious accidents and are valuable in helping to determine industrial safety performance at the station.
\item Utility personnel assigned to the station include all utility personnel, regardless of to whom they report. For example, a design engineer who reports to a manager at the corporate office, but is assigned to the station, would be included in accident rate statistics. Similarly, a utility electrician who normally does not work at the station, but is temporarily assigned to the station maintenance manager for a refuelling outage, would also be included in accident rate statistics.
\item The intention of this indicator is that it applies to accidents and injuries incurred in the course of company work by all assigned personnel. Accidents that occur off the employer's premises are considered work related if the employee is engaged in a work activity or if they occur in the work environment. Travel on company business would thus be considered a work-related activity.
Routine travel to and from work is not considered to be a work-related activity. One exception is necessary, however,  to establish an equitable basis for comparison in the case of employee fatalities or lost or restricted time injuries at stations that provide housing and other community services for off-duty employees and their families. Because most stations do not have such facilities, accidents or injuries incurred during their maintenance or operation by company employees are not to be included in the scope of this indicator.  In a similar manner, man-hours worked on such facilities should also be excluded from the indicator data reported to WANO.
\item For stations with both licensed units and units under construction, lost-time accidents and man-hours worked associated with units under construction are not included.
\item The normalisation factors of 200,000 or 1,000,000 man-hours worked are used for this indicator because several countries already collect and display lost-time accident rates using one of these factors.
\item An ‘accident’ is reported for each individual involved, not for a single event; for example, a single fire that caused lost-time injuries to three station personnel would count as ‘3’, not as ‘1’.
\item Reportable accidents that occur during business travel do not
  include those while on a business trip, a leg injury incurred during
  leisure activities, after business activities for the day were
  complete, would not count.
\end{itemize}

\subsubsection{Contractors}
\paragraph{Data Elements}
The following data is required:

The total number of accidents during the reporting period that
includes:
\begin{itemize}
\item Contractor restricted work accidents: known restricted work accidents involving days of restricted work for contracted, supplemental, or other non-utility personnel working onsite
\item Contractor Lost-time accidents: lost-time accidents involving days away from work for contracted, supplemental, or other non-utility personnel working onsite
\item Contractor work-related fatalities: the number of work-related fatalities for contracted, supplemental, or other non-utility personnel working onsite, not including illness
\item Total hours worked by contractors: the number of hours worked by
  contractor, supplemental, or other non-utility personnel working
  onsite during the reporting period
\end{itemize}
An example of the data collected for this performance indicator is
provided in Attachment \ref{CISA}.

\paragraph{Data Qualification Requirements}

An "accident" is reported for each individual involved, not for a
single event; for example, a single fire that caused lost-time
injuries to three station personnel would count as "3", not as "1".

Hours should be as accurate as possible based on available
data. Estimates of total hours are acceptable and can be reported to
the nearest 10,000 hours each quarterly reporting period.

A single accident should only be reported in the most serious of the
categories, with fatalities being the most serious and lost-time
accidents being next most serious. Thus, an accident resulting in lost
time which eventually resulted in a fatality would initially be
reported during the quarter it occurred as a lost-time accident, and
later updated to a fatality.

Any accident that results in a work-related fatality shall be included
as part of the indicator. Although not considered lost-time accidents,
these incidents represent the most serious accidents and are valuable
in helping to determine industrial safety performance at the station.

For stations with both licensed units and units under construction, accidents and hours worked associated with units under construction are not included until that unit begins commercial operation. For the purpose of this indicator, the construction zone of the new unit is not considered “onsite” of the operating station.
Due to difficulty in consistent and accurate reporting of accidents
and hours worked by non-utility personnel, the data for all the
non-utility personnel (or utility personnel not assigned to the
station) are limited to locations considered “onsite”.

\paragraph{Clarifying Notes for contracted, supplemental, and other
  non-utility personnel working onsite}

All non-utility personnel working onsite includes contracted,
supplemental, and other personnel working onsite. For clarity, this
may also include utility employees not assigned to the station but
working onsite.   Some countries include contractors as “supplemental
personnel”, but the intent of the indicator is to include hired, or
loaned, workers not directly employed by the utility, or contractors,
in addition to other non-utility personnel working onsite who may be
in a nonpaid status.

For the purpose of this indicator, “onsite” includes the utility owned
or operated station property or facility that is part of the operating
nuclear power plant or a facility directly supporting the power
station such as a training facility, storage area , and intake
structure. The property or facility may be at a different physical
location then the generating station. The intent of this indicator is
to focus on the operation of units at the station; therefore, for this
indicator, “onsite” does not include the physical construction zone of
a new unit on the station grounds until that unit begins commercial
operation. Therefore the accidents or hours worked during the
construction activities of a new unit are not included in reported
data.

Accidents and hours for visitors onsite are not reportable. For
clarity, non-industry visitors touring the power plant are not
included in this indicator; conversely, any people onsite in a work
status (i.e. to do work) are not to be considered as non-reportable
visitors for this indicator.

To be counted, an accident must result in one or more days away from
work (excluding the day of the accident) or result in a fatality.

Estimates of reportable hours of all non-utility personnel onsite may
be made based on contracted hours, historical data, plant access
security records, a combination of sources, and any other method that
provides a reasonable record or estimate of reportable
hours. Estimates can be reported to the nearest 10,000 hours each
quarterly reporting period.

A worker loaned by one utility to work at another utility’s station
during an outage is within the scope of this indicator. If that worker
sustains a restricted work, lost-time or fatal injury onsite, the
injury is reported. (The hours worked by all similarly loaned workers
are included in the work hours reported for the monitored station.)
If the worker was released from the loan on the day of the accident
and returned to his own utility, the extent of the injury would
determine if the accident would have resulted in restricted work or
lost time if the worker had stayed at the reporting station; if so,
the accident would count. The hours worked while the worker was onsite
would be included.

