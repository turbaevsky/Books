\subsection{Checklist for Administrative
and Technical Review of
WANO Performance Indicator Data Submittals}

Final (03 July 2009)

\subsubsection{INTRODUCTION}

The Performance Indicator Programme is an important vehicle for exchanging
information among WANO members regarding plant performance. When
usedcorrectly, it provides members with the ability to identify
othersthat are achieving better results and, through interaction with
these members, emulate the activities that are leading to the better
results.

For the Performance Indicator Programme to be successful, accurate and
complete data must be reported by all WANO members. WANO has developed a
Performance Indicator Reference Manual that defines each of the data elements
to be collected and how those data elements are used to calculate the
performance indicators. Personnel reporting and reviewing performance indicator
data should be familiar with the portions of the reference manual that relate to the
data they will report or review.

In many cases, inaccurate data reporting has been identified during peer reviews,
when training on the process is provided, or during detailed reviews of
performance indicator results. Many of these inaccuracies are the result of data
entry errors or inaccurate transference of information from one data source to the
WANO Data Entry System. In other cases, the errors are a result of incorrectly
applying the guidance provided Performance Indicator Reference
Manual. If there are questions regarding the manual, they should be checked against the
questions and answers on the web site. If the questions are not answered on the
web site, they should be directed to regional centre performance indicator
programme manager.

The WANO Governing Board has established a goal of 95\% of the WANO units
reporting correct data for all performance indicators. In order to meet this goal,
reviewers should ensure the data is complete and technically accurate. This
checklist will support actions to ensure the data is accurate and complete to
support calculation of performance indicators.

The checklist is organised based on a general review of the data, followed by
specific sections based on the data entry groupings in the WANO Data Entry
System on the web site.

\subsubsection{PURPOSE}

This checklist provides assistance for regional centre performance indicator
programme managers to confirm the data is reasonable when reviewing data
submitted by members. It is also available for use by member personnel that
enter or review performance indicator data for their station.

\paragraph{GENERAL DATA REVIEW}
\begin{itemize}
\item The data reviewer should verify the
data is reasonable and consistent with
other data provided to
WANO based on the information available to the
reviewer.
\item Ensure that all required data has
been entered. [Not entering a value in
required data element fields will prevent the dat
a category from being
submitted to WANO.
\item Use comment fields to add information that will be useful in understanding
the reported data.
\item For any text entered in the WANO or utility comment fields, ensure that
only Roman letters are entered. [Use of other characters may result in the
WANO Data Entry System “locking” and becoming inaccessible for that
unit.]
\item Confirm the ‘DN’ value is only
 used for Chemistry and Fuel Reliability data
fields where values are not appropriate because the unit was not at
required power levels for that data.
The ‘DN’ codes will prevent calculation
of the associated performance indicator.
\item The ‘X’ code can be used in any field but should only be used when the
required data was not collected even though it was technically available.
The ‘X’ codes will prevent the calculation of the associated performance
indicator.
\item When submitting new data, ensure that
 station-level data (emergency AC power and personnel safety data) is
 selected in addition to unit-level data during the data submittal process.
\item At the completion of data entry or submittal, consider using the Calculate
Indicators Report in the Data Entry System to see if the results are
reasonable. Unexpected indicator results may be an indication that
reported data is incorrect or missing.
Use the ‘View Data’ feature in the Data Entry System to review data
associated with the report results.
\item For all entries, check to ensure there are no apparent typographical errors
in entered data, such as an extra ‘0’ or an incorrect number of decimal
places.
\item Ensure data changes to previous reporting periods are updated AND
submitted as soon as possible. This provides WANO members with
accurate results. Comments are suggested for any significant data
revision that will affect calculated results. The comments should explain
why the data revision was needed and how error was identified.
\item Use the QRR in the Data Entry System to check all provided data.
Checklist for Administrative and Technical Review of WANO Performance Indicator Data Submittals
\end{itemize}

\paragraph{GENERATION DATA}
\begin{itemize}
\item  The sum of the energy loss values [Planned Energy Loss, Unplanned
Energy Loss (Forced), Unplanned Energy Loss (Outage Ext), and
Grid-related Energy Loss] should never exceed the Reference Energy
Generation for the period. If this occurs, the generation
data for the period will not be used in calculations. [NOTE: The
Reference Energy Generation value is automatically calculated by the
computer programme using the Reference Unit Power value multiplied by the number of
calendar hours in the quarter.]
\item If the Reference Unit Power is incorrect or needs updating (perhaps due to
a power uprate) inform the regional centre staff to ensure
the correct value is entered in the system.
\item If the number of critical hours reported in a period is less than the number
of calendar hours, check to see if generation loss data is reported and
whether the values appear correct. (For example, an automatic scram
would typically result in fewer critical hours and corresponding forced
generation losses. Similarly, fewer critical hours and outage extension
losses would be reported when a planned outage takes longer than was
originally scheduled.)
\item If an unplanned automatic scram is reported, unplanned energy loss
(forced) is expected unless the cause of the scram was due to causes not
under management control, such as most lightning strikes.
\item For data periods beginning in 2006, ensure that ‘0’ is entered in the grid-
related generation loss field if there are no grid-related energy losses to
report. If an area is known to have grid problems, the WANO reviewer
should monitor to ensure grid loss data is being reported.
\item Should comments be entered? It is a good practice to enter comments for
all scrams, significant energy losses, and critical hours less than calendar
hours. This makes it easier to confirm the reported data and can help later
users understand the values entered.
\end{itemize}

\paragraph{EQUIPMENT PERFORMANCE}
\begin{itemize}
\item Confirm the values entered for Fault Exposure Unavailable Hours are
accurate or appear appropriate.  Fault exposure is expected when
unplanned unavailability is reported and this reflects a discovered failure.
Fault exposure may be zero if discovered at the time of failure, one-half of
the periods between the discovery of the failure and the last successful
operation of the failed component if the start of the
failure is unknown, or the actual time since the failure star
t if the start of the failure can be determined. Fault exposure hours
may need to be entered for several reporting periods.
\item Check to ensure all unavailabi
lity hours for safety systems appearappropriate. Hours in any one period
shall not exceed the number of calendar hours in the peri
od. If there are no critical hours in the period, the
safety system data should still be reported, including ‘0.0’ for periods
where there is no reported unavailability.
\item Although the member cannot change the number of
 required trains per system, does the number appear appropriate and are unavailability hours
consistent and appropriate for that number of trains and the generation
data reported. Any proposed changes in the number of required trains
must be provided to the regional centre for review.
\end{itemize}

\paragraph{CHEMISTRY}
\begin{itemize}
\item Review data to identity any unusual or missing values. Only PWR molar
ratio data (S/G molar ratio target and actual values) may be submitted with
no data entered if the unit is not using molar ratio control. For PWRs on
molar ratio control, chloride data is not required to be entered.
\item Compare chemistry data to generation data to verify any ‘DN’ code is
appropriate. Also, compare the number of days greater than the required
power level to the number of critical hours andthe energy losses reported
in Generation. [There is no expectation that power has to be stabilised,
only that it is above the required power level when data is collected. Data
may be carried over from one reporting period to another.]
\item If the unit was shut down and/or below the required power level for the
entire period, ensure ‘DN’ is entered in the appropriate data fields.
\end{itemize}

\paragraph{RADIATION PROTECTION}
\begin{itemize}
\item Confirm the values entered are consistent with selected the unit of
measure.  [If man-Sieverts are selected and the numerical value entered
is a man-Rem value, the reported data used in calculations will be 100
times greater than the intended value. Similarly, if man-Sieverts is
selected but the value entered is in milli-Sieverts, the indicator result will
be much greater than what it should be.]
\item Normally, the External Whole Body Exposure will be greater than the
Calculated Internal Whole Body Exposure. Ensure a value (usually zero)
is entered in the Calculated Internal Whole Body Exposure field.
\item The period data should be consistent with critical hours and generation
data. At multi-unit stations, data should be reported separately by unit if
practical. If one unit was in an outage, the reported values should be
significantly higher than the other unit. If all units at the site operated
throughout the entire reporting period, evenly dividing exposure among
the units is acceptable.
\item Exposure data is usually updated subsequent to the reporting period due
to obtaining more accurate data. The data should be updated and
processed as soon as possible.
\end{itemize}

\paragraph{FUEL RELIABILITY (data reported on a monthly basis)}
\begin{itemize}
\item Confirm that the numerical values reported are consistent
 with the units of measure selected in the Data Entr
y System for the reactor type.
\item Confirm fuel data is consistent with chemistry and generation data.
\item For applicable reactor types, confirm that ‘DN’ is entered for months when
the reactor is not critical or when minimum power requirements are not
met during the month. Compare the month’s average power level with
energy losses reported in Generation data.
\item For applicable reactor types, confirm that the percentage
of reactor power is entered for Power Level for Activity Measurement (\%). This value must
be 85\% or greater to use the monthly data in the quarterly calculation. If at
least one valid month is not available, the programme for calculating the
performance indicator uses the next most recent quarter.
\end{itemize}

\paragraph{PERSONNEL SAFETY}
\begin{itemize}
\item Compare the number of critical hours and generation losses reported in
Generation data with the number of hours worked by station personnel
and contractors. Total hours worked by station personnel and contractors
are typically greater during outage periods.
\item The hours worked is not expected to be zero and some contractors are
expected to be working on site or for the station. If ‘0’ contractor hours are
reported, compare this to previous quarters. If there is a difference, a
comment should be included to explain why the number of contractor
hours has significantly changed.
\item The number of restricted work accidents over a long period is usually
higher than the number of lost time accidents.
\item Accidents should be reported in a ccordance with the WANO reporting
manual and this may not follow local government or regulatory guidelines.
\item Data updates are not unusual and should be reported and processed as
soon as possible. For example, a restricted work accident may need to be
changed to a lost time accident and a lost time accident may need to be
changed to a fatality. These updates should be made in the period when
the accident occurred, with only a change in the accident category.
\end{itemize}