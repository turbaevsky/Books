\section{User Manual}
\subsection{WANO Data Entry System (DES)}

\subsubsection{Overall Description}

The WANO Data Entry System (DES) is a Web-based application developed and implemented for WANO member units to directly enter performance indicator data reportable to the WANO Performance Indicator Programme.

The reported data is used by WANO to calculate performance indicator values.  These results are periodically posted on the WANO Web site. Results are available on spreadsheets and in Web-based PI REPORTS, both available on the WANO Web site.

DES has a Home Page that provides various programme information, reminders, and a link to Current DES Issues. These all should be reviewed periodically.

\subsubsection{Access}

The application requires access to the Internet using Internet Explorer version 5.0, service pack B, or better. If access is not currently available, a member may request its regional centres to enter its data for its units; however, it is strongly suggested that each member work towards improving its Internet access to benefit from efficient data reporting and utilization of the system features.

The system is accessed through the Performance Indicator section on the WANO Web site.  An ID and password are required to access the WANO Web site; in addition, specific authorizations must be made for the ID to perform various actions within DES. The unit’s assigned local administrator for the performance indicator program controls those authorizations.  The WANO regional centre data coordinators initially assign the local administrator authorization. The system administrator is located in WANO-Atlanta Center. (Passwords expire every 180 days.)

\subsubsection{Personal Authorization}

Local administrators control access to DES once an ID is assigned to the individual seeking access.

\subsubsection{Features}

The following features are included in DES:  Data Entry (the capability to enter data into data categories); Data Submittal (the capability to submit data after entry to WANO regional centres for review, acceptance, and processing); Data Review (the capability for regional centres to review and process data submitted by members); Data View (the capability to view data entered into DES); and the capability to run several types of reports.

The Reports include data quality reports, calculations, timeliness status, and unit pedigree information.

The system also allows local administrators flexibility in assigning authorizations to other IDs of the same organization.  Security administration can be accessed under the Maintenance tab on the DES Home Page.

Several other features are included under the Maintenance tab that allows certain label customizations and selections. These features are suggested for advanced users.  Contact the regional coordinator regarding these features.

Help screens are available under the Help tab but are very basic and limited. Improvements will be made as resources permit. Although print and copy capability is possible, the system is designed for electronic entry, submittal, and viewing of data, and therefore printouts may not be suitable for formal reports.

Users should review all available tabs and menu items to identify all available capabilities.

\subsubsection{Use}

The basic use of DES is as follows:

Authorized IDs enter PI data. Data can be entered, saved, viewed, and calculated prior to submittal to WANO. Data may even be entered “in advance” – for indicator projection calculations. (However, data cannot be submitted prior to the end of the period.)  Data entered at the unit/utility level has a green background.

Data can and should be reviewed by other utility staff prior to submittal to WANO.  The data submittal authority is a separate authorization from data entry.  Furthermore, while data entry authority may be assigned to IDs by category, the assignment of submittal authority is for all data categories, although the submitter may elect to submit data categories at different times.  It is recommended that initial data submittals for a new reporting period be done for all categories at the same time. The station level submittals are separate from unit level submittals--submitters should not forget that both submittals need to be done each period. Data that is submitted for WANO review has a yellow background.

IDs with data entry privileges may also Recall data that has been submitted--returning the data from the review (yellow) status to the utility level (green).

After data is submitted (a due date has been established for each reporting period--please check with your regional coordinator for the due date), the applicable WANO regional centre will review the submittals. The various results from review can be accepted, or returned to the submitter electronically. The reviewer may also elect to phone the submitter to resolve any questions. If the data is accepted, the reviewer will promote the data to production. The data will then be used in WANO calculations when the WANO system administrator calculates results. Data that is promoted (in production) has a white background.

Data can also be disqualified--a UD code is assigned to a data element and then has a red background--or the data may be returned to the utility (the data is returned to a utility level (green) from the review status (yellow).

Results are typically calculated by WANO once a calendar quarter after all worldwide data is placed in production.

Authorized IDs may calculate any indicator’s value in DES for the associated unit using data in the various levels (green, yellow, or white backgrounds.)

Users with data entry, data submittal and view data authority may calculate indicator values for the associated units/stations.

The QRR (Quality Review Report) may be used to identify potential data errors.

All data fields have associated comment fields. The comment fields are
text fields that can contain up to 255 characters. Only Roman style
characters should be used. It is recommended that comments be provided
for unusual or significant data or data changes. WANO reviewers have a
separate comment field for WANO comments. Utility and WANO comments
are separate and cannot be modified by each other.

\subsubsection{Data Entry }

Authorized users enter performance data or codes in specific categories and the related comments if appropriate. If the category field has a selectable option for the corresponding unit of measure, that selection MUST be made first, turning the data field color green, before entering the numerical value. Data with a green field is called utility data as it has not been submitted to WANO for review. Calculations can be made in DES using utility data.

\subsubsection{Submittal}

Authorized users submit completed performance categories for specific periods.  Any or all categories may be selected for single or multiple periods at a single time. After submittal, the data field turns yellow. Data with a yellow field is called Review level data and is available to WANO for Review and acceptance. Data at the Review level cannot be changed by the utility. It is also not used by WANO in published calculations.   Authorized users can recall (by category) data in Review to utility data level if the data needs to be changed. Calculations can be made in DES using Review level data.

A separate ID can be used to submit data in order to allow an independent verification of data quality before data categories are submittal to WANO.

\subsubsection{Review by WANO Regional Staff}

Authorized users (IDs having Submit authority) submit completed performance categories for specific periods.  Any or all categories may be selected for single or multiple periods at a single time. After submittal, the data field turns yellow. Data with a yellow field is called Review level data and is available to WANO regional staff for Review and acceptance.  Data at the Review level cannot be changed by the utility. It is also not used by WANO in published calculations.  Authorized users can recall (by category) data in Review to utility data level if the data needs to be changed. Calculations can be made in DES using Review level data.

\subsubsection{Calculation of Results }

Under the Reports tab, a user with at least View Data authority can calculate results and display those values in one of six formats.  Depending on the criteria selected, the user can calculate results based on utility level (green) data only, or production level (white) data only, or production level data with the Review level data (yellow) where it is present, for one or more units, and one or more data periods, covering one or several time periods. Please contact your regional coordinator for more details.

\subsubsection{Other Features }

The regional coordinator should be contacted regarding other features that may be available or other issues not addressed in this document. The DES Home Page may also be referenced for the latest information on the status of DES and current DES issues.


\subsection{PI Reports}
\subsubsection{Intoduction}

This moment (\today) PI Report system has all the precalculated
PI results excepting TISA but cannot create the most important reports
like Long-term Targets, Trifold (annual summary), spreadsheets etc.

Currentrly all the extra reports are provided locally by London
office.

LO has a plan to integrate all te necessary reports into updated PI
Report system, but is seems available only after 2020.

This moment all the extra reports might be requested by members and RC
to WANO LO.

When all the source data has been received from RCs the PI System
Administrator (SA) should initiate calculation (using Maintenance tab in
the PI Report system) then release results using the same tab there.

From this time the system will update (if any) and publish the updated
results automatically every weekend. The SA also can initiate the
updating process manually at any time.

Members and RCs have not any limitation to chenge their source data at
any time for any period of time.