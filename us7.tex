\subsection{Unplanned Total Scrams Per 7,000 Hours Critical (US7)}

\subsubsection{Purpose}

The purpose of this indicator is to monitor performance of the number
of unplanned reactor shutdowns. The indicator provides an indication
of success in improving plant safety by reducing undesirable,
unplanned thermal-hydraulic and reactivity transients that require
automatic or manual reactor scrams. It also provides an indication of
how well a plant is operated and maintained. 

The inclusion of manual scrams in this indicator, is not intended to
discourage operator-initiated scrams and actions to protect
equipment. Rather, the total number of scrams, regardless of
initiation, required to shut down the plant due to undesirable,
unplanned thermal-hydraulic and reactivity transients provide an
indication of plant performance.

Taking into account the number of hours that a plant is critical,
provides an indication of the effectiveness of scram reduction efforts
while a unit is in an operating condition. In addition, normalising
individual unit scram data to a common standard (7,000 hours
critical), provides a uniform basis for comparisons among individual
units and with industry values.

\subsubsection{Definition}

The indicator is defined as the sum of the number of unplanned
automatic scrams (reactor protection system logic actuations) and
unplanned manual scrams that occur per 7,000 hours of critical
operation.
 
The value of 7,000 hours is representative of the critical hours of
operation during a year for most plants. It provides an indicator
value that typically approximates the actual number of scrams
occurring during the year.
 
\subsubsection{Calculations}

The unit and industry values for this indicator are determined for a period as shown below:
$$ \text{Value for a unit} = \frac{(AS+MS) \cdot 7000}{\text{total number of hours critical}} $$
Where   

AS = number of unplanned automatic scrams 

MS = number of unplanned manual scrams

$$ \text{Value for the industry} = \text{mean of the unit values} $$

The typical result for both an individual unit and the industry will not be an integer, because these calculations are based on the number of scrams resulting per 7,000 critical hours. For comparisons of individual units, unplanned scrams per 7,000 hours critical will be presented for a three-year period, to minimise the effects of variations in the indicator value during shorter time periods due to the low number of scrams at most plants. Units must average at least 1,000 critical hours per year to be calculated and be included in the worldwide (mean) value.
An example of the data collected for this performance indicator and a
sample calculation are provided in the Attachment \ref{us7}.

