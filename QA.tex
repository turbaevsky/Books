
\section{Performance Indicator Questions and Answers}

This section will be updated on a regular basis to address questions raised by individuals involved with WANO Performance Indicators. If you have a question that is not covered on this page, contact your regional PI coordinator or a member of the Performance Indicator Working Group (PIWG).

\subsection{Data Entry System Entries}

\paragraph{When should data entry codes DN and X be used instead of numerical values?}

Data Entry Code Entries

    The data entry codes DN and X should only be used in the following specific circumstances. Normally, numerical values are entered and data fields are never left blank.

    The DN code should only be used instead of a numerical value when the value cannot be determined due to power requirements of the reporting guidance for the data element. Only the chemistry and fuel reliability data categories have data elements that have power requirements. Therefore, if no data element value meeting the data requirements is available, a DN code should be entered instead of leaving the field blank.

    The X code is used in any data element field where the value is not available due to circumstances that prevented collection of numerical data that would normally be available.

    The use of DN or X in a data field results in that period's data elements not being used in indicator calculations.

\subsection{Performance Indicator Results Spreadsheets}

\paragraph{Several columns in the unit performance indicator result
    published on the WANO Web site contain the words "data code" in
    the title. What are these columns for, and what do the single
    characters in these columns represent?}

Performance Indicator Results Spreadsheets

    Several columns in the unit performance indicator results published on the WANO Web site contain the words "data code" in the title. What are these columns for, and what do the single characters in these columns represent?

    These columns indicate whether valid results could be calculated for the associated indicator using the raw data provided by the station. The letters are data disqualification codes. The presence of a disqualification code usually means that a valid result could not be calculated, and the letter indicates the reason why. In the WANO year-end report, indicator values with a disqualification code are excluded from calculation of industry values.
\begin{tabular}{p{0.5cm}p{14cm}}
A & Commercial Date disqualification. The unit has no assigned
    commercial date or the indicator period starts before the unit's
    commercial date and there is:
    \begin{itemize}
    \item More than 2 months qualified data for a quarterly value, or
    \item More than 6 months qualified data for a 1-year value
    \end{itemize}\\
B & Commercial Date disqualification. The unit has no assigned commercial date or the indicator period starts before the unit's commercial date and there is:
\begin{itemize}
\item More than 12 months qualified data for a 2-year value, or
\item More than 18 months qualified data for a 3-year value.
\end{itemize}\\
C & Commercial Date disqualification. The unit has no assigned commercial date or the indicator period starts before the unit's commercial date and there is
\begin{itemize}
\item Less than 2 months qualified data for a quarterly value, or
\item Less than 6 months qualified data for a 1-year value, or
\item Less than 12 months qualified data for a 2-year value, or
\item Less than 18 months qualified data for a 3-year value.
\end{itemize}\\
D & Value not qualified due to errors in the data provided\\
H & Value not qualified due to insufficient critical hours\\
L & Value not qualified because unit is in long-term shutdown\\
M & Value not qualified due to insufficient or invalid data\\
P & Value not qualified because reactor power was below required
    level\\
\end{tabular}

\paragraph{What do the column headings SP1, SP2, and SP5 mean in the
  performance indicator results table published on the WANO Web site?}

These headings identify the safety systems monitored by the safety system performance indicator. SP5 refers to the emergency AC power system. Other systems monitored vary according to reactor type. The table below explains what systems are referred to by the designators SP1 and SP2.

\begin{tabular}{p{4cm}p{5cm}p{5cm}}
Reactor Type & SP1 & SP2 \\
\hline
PWR / VVER & high pressure safety injection system & auxiliary
                                                     feedwater system \\
BWR & high pressure injection/heat removal systems & residual heat
                                                     removal system \\
RBMK (first generation) & emergency heat removal system &\\
RBMK (second generation) & first emergency heat removal system &
                                                                 second emergency heat removal system \\
PHWR & high pressure emergency coolant injection system & auxiliary
                                                          boiler
                                                          feedwater
                                                          system \\
Magnox Reactors & emergency feed system & tertiary feed system \\
Advanced Gas Cooled Reactors & emergency feed system &back up cooling
                                                       system/decay
                                                       heat loops \\
FPF (fuel processing facility) & N/A & N/A \\
\end{tabular}

\subsection{Performance Indicator Definition Interpretation Issues}
\begin{longtable}{p{7cm}p{7cm}}
    In the Performance Indicator Reference Manual chapter for
  "UNPLANNED SCRAMS PER 7000 HOURS CRITICAL", a clarifying note states
  : "Each scram caused by intentional manual tripping of the turbine
  should be analyzed to determine those which clearly involve a
  conscious decision by the operator to manually trip the turbine to
  protect important equipment or to minimize the effects of a
  transient. Scrams that involve such a decision are considered to be
  manual scrams." However, is the scram reported as a manual scram or
  automatic scram when the reactor scram is caused by the operator
  manually tripping the turbine unintentionally or unconsciously
  because of human error? &
    If an operator had no intention (did not involve a conscious
                            decision) of tripping the turbine to
                            protect important equipment or to minimize
                            the effects of a transient (whether by
                            human error or otherwise) and an automatic
                            scram resulted from the operator actions,
                            the reactor scram would be reported as an
                            automatic scram. The clarifying note in
                            the guidance is provided to characterize
                            the intentional turbine trip that results
                            in a reactor scram as "manual" because the
                            intention of the operator was to protect
                            important equipment or to minimize the
                            effects of a transient by tripping the
                            turbine and the reactor (in other words,
                            it was a conscious decision). When an
                            unintentional trip of the turbine results
                            in a reactor scram, it is reported as
                            "automatic" because it reflects the
                            response of the reactor protection system
                            to the unintentional turbine trip\\


    In the Safety System Performance Indicator chapters, a clarifying
  note is provided under Support System Unavailability that reads:
  "Unavailable hours are also reported for the unavailability of
  support systems that maintain required environmental conditions in
  rooms in which monitored safety system components are located if the
  absence of those conditions is determined to have rendered a
  monitored function of a train unavailable for service at a time it
  was required to be available." Since it is the unavailability of the
  function of the monitored system that is being reported, how does
  support system unavailability affect data reporting? If the
  temperature is normal, but cooling is unavailable, is unavailability
  reported? &
    Unavailable hours are reported for the monitored system, not the
              support system, so the reported unavailable hours are
              only those for when the monitored system is
              unavailable. If the temperature (or other environmental
              conditions) in a room are such (e.g., high), due to the
              unavailability of the support system, that the monitored
              train, as determined by station staff, cannot then
              perform its monitored function, unavailable hours are
              reported. If the temperature (or other environmental
              conditions) in a room are such (e.g., normal) that the
              conditions do not then affect the monitored function, as
              determined by engineering evaluation, no unavailable
              hours are reportable - even if the support system is
              unavailable - as long as the monitored train is able to
              perform its monitored function. Once environmental
              conditions reach the point the monitored train cannot or
              did not perform its function, unavailable hours are
              reported. An engineering determination is required to
              determine if (or when) the monitored function is lost\\

    A nuclear power station often does not own the switchyard even
  though it is on-site. Distribution company staff often perform work
  and various other activities in the switchyard. The station does not
  own the equipment and the workers may not even be assigned to the
  station. Are generation losses due to causes, actions, or equipment
  failures within the switchyard reportable to WANO for use in the
  UCF, UCLF, and FLR indicators? &
    WANO reporting guidance does not base reportability of generation
                                   losses on "who owns" the equipment
                                   or "who performed" activities that
                                   resulted in generation
                                   losses. Rather, reportability is
                                   based on whether the causes of the
                                   generation loss were "under the
                                   control of plant management.
Management control at a nuclear power station and its switchyard
  is viewed as very extensive. Because plant management is expected to
  "have control" over the switchyard equipment, the status of the
  equipment, and the activities in the switchyard that may result in
  lost generation, the generation losses would be reportable\\

    Multi-unit CANDU nuclear power stations have sometimes experienced
  difficulty in determining reportable component unavailability to
  WANO in the monitored safety systems due to the design redundancy
  and spares. This is partly due to the design and the difficulty in
  identifying the trains and the effect on the identified trains. How
  can local terminology assist in determining the data to be reported?
  How is the indicator calculated for these stations when as little as
  one train may serve multiple units? &
    The WANO performance indicator program reference manual establishes guidance for reporting unavailability data for three separate safety system performance indicators. For comparability worldwide, the indicators are defined in terms on "unavailability per train". The indicator is calculated using each system's number of trains, pre-identified by the station, and "unavailable hours" based on "component" unavailability that result in the monitored train/system being unable to perform the monitored function, and the number of hours these trains are required to be available for service while the unit is operating. Since CANDU reactors at multi-unit stations often have (are designed with) component redundancy built-in rather than a redundancy of trains of components within a system, the provided guidance may be difficult to apply consistently.
    At some stations, only one monitored train may exist with multiple redundant components within the train. Determining "unavailable hours" requires knowledge of the components to be monitored (only those required to be in service while a unit is operating) and the hours a system/train is required to be in service. The hours those trains are required to be in service are defined by the guidance depending on the monitored system, and a default value is used in the indicator calculations. The unavailable hours are reported by every unit that is affected by the unavailability.
    At many CANDU units, component unavailability affecting system function is defined in terms of "levels of impairment". Multiple pumps and other components result in train/system redundancy that allows an unavailability of some components to have no affect on the ability of the monitored system to perform the monitored functions. These redundant components may be considered installed spares if unavailability does not affect the ability of the trains to perform its functions. These may be currently identified as "level 3" impairments, i.e. reduction in redundancy. An unavailability that results in a "level 1 or 2 impairment" is considered as resulting in component unavailable hours, and those unavailable hours (planned or unplanned) are reported to WANO. Identified fault exposure hours must be considered (unavailable hours of "spares" are not reported, but the full reference manual guidance should be reviewed for details). As mentioned, if an unavailability affects more then one unit at a multi-unit station, the unavailability is reported by each unit.
    Knowing unavailable hours of the monitored components, the required hours (default values), and the number of trains, the safety system indicators can be calculated in terms of "unavailability/train" for each of the three systems within the WANO PI program\\

    At a PWR, letdown flow normally includes flow through
  demineralizers. If the demineralizers are out of service the iodine
  concentrations in the reactor coolant increase. Does having the
  demineralizers out of service satisfy the steady state conditions
  needed for valid data collection? In other words, should fuel
  reliability indicator (FRI) data be collected and used from periods
  when the purification demineralizers are out of service?
& For PWRs, it is assumed that data collection occurs with normal
  letdown (purification) system conditions existing; these conditions
  include the demineralizers being in service with normal letdown flow
  and steady state chemistry conditions. If these conditions are not
  met the FRI data collected while the demineralizers were out of
  service should not be included with the remaining monthly data used
  to determine the FRI data element values for that month. This is the
  same as not using the data from when the unit is below 85\% power or
  the unit was not at steady state power for at least three days as
  defined in the guidance\\

    Regarding the industrial safety accident rate indicator (ISA), if
  an employee who normally performs five different tasks is injured
  such that he/she is no longer capable of performing one of those
  tasks (but still capable of performing the remaining four tasks),
  should that injury be reported as a restricted work accident? (This
  also applies to contactor employees for CISA.)
& Yes. Even though the employee is capable of performing most of the
  tasks, since he/she is not capable of performing them all, it is a
  restricted work accident\\

    Regarding the safety system performance indicators (SSPI), I
  understand that some other non-WANO indicators do not use the same
  guidance for fault exposure as WANO, and some organizations do not
  wish to include fault exposure hours in their calculations; are
  fault exposure hours still reportable to WANO for the WANO
  performance indicators?
& Yes. Fault exposure hours are still to be reported to WANO as part
  of the safety system performance indicator as specified in the
  guidance. There are several (non-WANO) indicators throughout the
  worldwide nuclear industry and their definitions may vary from the
  WANO definitions. However, the WANO safety system performance
  indicators use fault exposure hours as one of three unavailability
  terms in its calculation and therefore the fault exposure data for
  all reportable systems are to be reported to WANO as part of the
  WANO performance indicators\\



    We have heard that some non-WANO groups asked that fault exposure
  data collection be stopped. Is that true?
& Yes, a non-WANO member did request that the fault exposure data not
  be collected for emergency AC power systems. Since fault exposure is
  considered an integral part of the determination of all safety
  system unavailability, fault exposure was not removed from the
  indicator definition and data collection of fault exposure hours was
  never stopped. All fault exposure hours are to be reported per the
  WANO reporting guidelines for all the reportable safety systems,
  including emergency AC power\\


    What planned outage end date do I use to determine the amount of
  unplanned energy losses that my unit experiences? Our station has an
  end date established with and agreed to by the grid dispatcher, but
  we also have a work schedule which shows a scheduled outage end
  date. Some units have a detailed schedule of activities used for
  directing the outage. Which date is the basis for determining if an
  outage extends beyond its planned end and therefore requires
  "unplanned energy losses (outage extension)" be reported?
& The clarifying notes of the reporting guidance specify that the end
  date of planned outages are those negotiated with and agreed to by
  the network and/or grid dispatcher. This date may differ from dates
  shown on schedules used by the utility to manage the outage on a
  day-to-day basis\\

    There is a small amount of dose received outside of our generating
  stations (waste reduction, laundry facilities, radiography) that
  originates from station radioactive or contaminated material. We
  currently take this dose and divide by the total number of stations
  and add the result to each unit's total whole body dose. Is it
  appropriate to include this nonstation dose in the CRE?
& Based on the indicator definition, the indicator is measuring
  exposure "at the facility". Even though the material originated
  on-site, only (and all) exposure to monitored personnel on-site
  should be counted. Nonstation dose should not be counted\\

    Regarding Standby Emergency AC generators (2 out of 4 required),
  is a station penalized for the unavailability of one or two
  generators if only two are required to meet the design criteria?
& The unavailability incurred for any one or two generators when
  multiple emergency AC power trains are installed depends upon the
  design basis requirements at a station. To not have unavailable
  hours of a train count, or be reported, the emergency AC power train
  must either be "not required" or must be an installed spare that is
  not in service. Whether the plant is operating or shutdown, the
  train is considered to be required and any unavailable hours counted
  (and thus, are to be reported).  An exception for a single- or
  multi-unit station with all units shut down, is that one emergency
  generator at a time may be electively taken out of service without
  incurring planned or unplanned unavailable hours, providing that at
  least one operable emergency generator is available to supply
  emergency loads.
    An installed spare train is one that is used as a replacement for
  other trains to allow for the removal of equipment from service for
  preventive or corrective maintenance without incurring a limiting
  condition for operation (where applicable), or violating the single
  failure criterion. To be an "installed spare", it must not be
  required in the design basis safety analysis for the system to
  perform its safety function (and obviously, not acting as a
  replacement when the unavailability occurs). Additionally, some
  stations using the train failure-based reporting option may be able
  to take credit for extra (redundant) trains to avoid needing to
  count certain unavailable hours. See the train failure-based safety
  system performance indicator definition for further details.
    Also regarding a required train, as specified in the reporting
  guidelines, "unavailable hours are recorded only when the emergency
  generator train is unavailable to deliver emergency AC power. The
  failure of one of two redundant emergency generator support
  subsystems, for example, would not count toward emergency generator
  unavailability as long as the emergency generator train was still
  available."\\

    It is not clear when reporting Industrial Safety Accident Rate
  information if the number of hours is payroll hours (including
  vacations, sick leave, and other absences), or just hours worked.
& The reporting guidance states that the number of hours is to be the
  number worked, not payroll hours\\

    Why does WANO still require the reporting of fault exposure
  unavailable hours as a part of the safety system performance
  indicator unavailability data?
& Without the inclusion of a term to account for the effect on system
  unavailability of undiscovered faults, the indicator would be
  nothing more than a maintenance indicator showing the effects of
  planned and unplanned maintenance time. The intent of the indicator
  is to depict all sources of unavailability when the system is
  required to be in service. Sometimes, an undiscovered failure can
  render a train or system unavailable for prolonged periods of time,
  greatly effecting the overall system unavailability. The
  probabilistic approach used to estimate the unavailable hours due to
  latent or undiscovered faults is sound. More frequent testing of
  systems with weak performance histories can enable the detection of
  latent failures sooner which, in turn, reduces the unavailable hours
  due to those failures\\

    Since the mathematical approach to calculating fault exposure
  unavailable hours is essentially probabilistic in nature, shouldn't
  the individual unit and industry SSPI goals reflect what PSAs allow
  regarding system unavailability?
& The safety system performance indicator was never intended to
  produce data directly usable in PSA calculations, nor were the
  established SSPI industry goals intended to be directly tied to
  individual station PSA assumptions. In setting year 2000 performance
  goals, many stations checked their PSA assumptions to ensure they
  did not establish a goal outside of what the PSA assumed, but in
  many cases, performance of these systems was already better than
  what PSAs assumed. Therefore, the goals set by individual stations
  for 2000 were based on past performance, modified, perhaps, by
  predictable effects of increased online maintenance or other
  factors\\

    The Performance Indicator Programme has been in place for many
  years and the original ten indicators included thermal performance
  and radioactive waste volumes. Also, the WANO Results Spreadsheets
  for 1997 and earlier include this data. Where can the earlier
  performance indicator definitions be found?
& The definitions for the early performance indicators (those in use
  before 2001) can be obtained from the Performance Indicator
  programme manager in the Coordinating Centre. The early performance
  indicator definitions were modified and replaced in January 2001
  with ten indicators. Over time, additional changes have been made
  and these are shown in the current version of the Performance
  Indicator Reference Manual\\
\end{longtable}

Rev (CC) 24/05/12
